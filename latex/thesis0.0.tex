\documentclass[a4paper,10pt]{report}
\usepackage[utf8]{inputenc}
\input{/home/andi/Studium/LaTex/preambel.tex}


%opening
\title{Magnetic Excitations In Iridiates}
\author{Andreas Karl Leonhardt}

\begin{document}

%\maketitle

%\begin{abstract}

%\end{abstract}
\tableofcontents

\chapter{Introduction}
\section{Towards A Spin Hamiltonian}

\subsection{Iridium And Compounds}
Iridium is a transition metal, i.e. it has a only partially filled $d$ shell, the shell structure of atomar Ir is given by $[\mathrm{Xe}]4f^{14}5d^7 6s^2$.
% ${\mathrm{Sr}}_{2}{\mathrm{IrO}}_{4}$
In the compounds treated in this thesis Iridium is oxidized to $\mathrm{Ir}^{4+}$ 
removing the $6s^2$ electrons as well as two electrons from the $5d$ shell. 
the ion has therefore $5d^5$ as the outermost and active shell ~\cite{Abragam70}.

%(the ion shell structure does not follow the Aufbau principle. 
%Therefore has $\mathrm{Ir}^{4+}$ a different shell structure than Tantalum, even though they have the same number of electrons. 
%This is because of the different charge of the nucleus, that leads to a greater seperation in energs levels with different quantum number n ($\propto \frac{Z}{n^2}$)).

The other elements of the compunds treated in this thesis have closed shells an are therefore spin free. In the case of $\mathrm{Ir}^{4+}$, 
$\mathrm{O}^{2-}$ has only closed shells ($1s^2,2s^2,2p^6$) and therefore is $L=S=0$.
the same holds for $\mathrm{Sr}^{2+}$ with the electron configuration of $\mathrm{Kr}$

\subsection{Ligand Field}

Embedded in the crystal of the compound, total rotational symmetry is broken and reduces to discrete symmetries about certain axes. 
This splits up the former degenerate $5d$ levels. 
In all of the compunds lookes at here, the $\mathrm{Ir}$ atoms are surrounded by octahedral of double negatively charged oxygen ions, the so calles ligands. 
Those octahedra might share corners or even edges, while the latter one simply means they share two edges. 
%
% In ${\mathrm{Sr}}_{2}{\mathrm{IrO}}_{4}$ the $\mathrm{Ir}^{4+}$ is surrounded by 8 $\mathrm{O}^{2-}$, building a octaeder with shared corners.
%
The field generated by those ligands split up the $5d$ levels in the threefold $t_{2g}$ and the twofold $e_g$ states. 
Since the ligand field is quite strong
\todo{quote, strong compared to what, Hund's rule and why it is violated}
only the $t_{2g}$ levels are populated in the ground state, leaving one hole in this band. 
The $e_g$ orbitals are not occupied in the ground state, since the crystal field splitting is stronger than the repulsion of electrons beeing in the same orbital. 
This leads to the so calles low-spin state, in contrast to a weak crystal field, 
where it would be more favourable to follow Hund's rules, 
i.e. filling up the $t_{2g}$ and the $e_g$ orbitals with electrons of the same spin, before putting to electrons with opposite spin into an orbital.
The $t_{2g}$ levels build an \emph{effective} $L=1$ system, since it is threefold degenerate.

\subsection{Strong Spin-Orbit Coupling}

In the $5d$ elements spin orbit coupling (SOC) plays an important role since it increases with the charge of the nucleus.
\todo{theoretical background for string SOC according to Z.}
Since there is one hole in the $t_{2g}$ shell left, the total Spin is $S=\frac12$. 
When coupled to the effective spin $L=1$ as mentioned above, 
the sixfold degenerate $t_{2g}$ states (including spin degeneracy) split into a fourfold degenereate level with $J=\frac32$
and twofold degenerate level with $J=\frac12$. The latter one is higher in energy, since we couple an \emph{effective} angular momentum to the spin,
thus the paralell coupling is energetically favourable.
In the ground state the $J=\frac32$ band will be filled, while the $J=\frac12$ band is half filled, leaving us with an effective Spin$\frac12$ system on each Iridium site in the grid.
\todo{create/include graph similar to \cite{PhysRevLett.101.076402}}

The two states are given by a linear combination of the molecular orbits and spin states,
\begin{equation}
 \ket{J_{\mathrm{eff}} =\frac12, M_{J_{\mathrm{eff}}}= \pm \frac12}
 = 
 \frac{1}{\sqrt{3}} \left( \ket{yz,\pm \sigma} \mp \im \ket{zx,\pm \sigma} \mp \ket{xy,\mp \sigma} \right).
\end{equation}
where $+\sigma = \uparrow, -\sigma = \downarrow$. The moelcular orbits are given by linear combinations of the spherical harmonics
\begin{IEEEeqnarray}{rCl}
\ket{xz} &=& \frac{1}{\sqrt{2}} \left( Y_2^{-1} - Y_2^{1} \right) \\
\ket{yz} &=& \frac{\im}{\sqrt{2}} \left( Y_2^{-1} + Y_2^{1} \right) \\
\ket{xy} &=& \frac{\im}{\sqrt{2}} \left( Y_2^{-2} - Y_2^{2} \right) 
\end{IEEEeqnarray}

\todo{transformation properties of the ground state}


\subsection{Spin System And The Hubbard Model}

\subsubsection{Tight Binding Model} % in second quantization




Energy eigenstates of the crystal must be Bloch states, defined by  the relation
\begin{equation}
 \Psi_{n,\vec{k}} \left( \vec{r}+\vec{t}\right) = \euler^{\im \vec{t} \vec{k} } u_{n,\vec{k}} \left( \vec{r} \right), \label{BlochDef}
\end{equation}
where $\vec{t}$ denotes a translation vector of the lattice, e.g. one that leaves the lattice unchanged, 
and $u_{n,\vec{k}}$ is a periodic function with the same periodicity than the lattice.
Using a Fourier transform on the Bloch states of different bands $n$, we get the so called Wannier states,
\begin{equation}
 \Psi_{\vec{R}_i, n} \left( \vec{r} \right) = \frac1{\sqrt{N}} \sum_{\vec{k}} \euler^{\im \vec{k}\vec{R}_i } \Psi_{\vec{k}, n} 
\end{equation}
Those are no longer eigenstates of the Hamiltonian, but still orthonormal, since the transformation is unitarian. 
Furthermore they are localized around the positions $\vec{R}_i$.
Instead of the eigenfunctions of the single particle Hamiltonian, we can use linear combinations of them. 
This gives us a different set if Wannier functions, that are related to the ones previously defined by an unitary transformation.
The additional degrees of freedom can be used to optimize the Wannier functions according to certain criteria.
The most common ones are maximal localization or symmetries of the crystal or the atomic orbitals.



For atoms with a very large distance in between them the Wannier functions approach states of free atoms or linear combinations of them.
Each band corresponds to a atomic orbital or a set set of degenerate orbitals.
If the overlap of the potentials of the atoms building up the crystal is small, we might use atomic states as a starting point for our description of the system.
We can construct a Bloch function by
\begin{equation}
 \Psi_k(\vec{r}) = \sum_{\vec{R}} \euler^{\im \vec{k}\vec{R} }  \phi_{n}(\vec{r}-\vec{R}). 
\end{equation}
This construction fullfills the requirement of \ref{BlochDef}, since
\begin{equation}
 \Psi_{n,k}(\vec{r}+\vec{t}) 
 = \euler^{\im \vec{k} \vec{t} } \sum_{\vec{R}} \euler^{\im \vec{k} (\vec{R}-\vec{t})} \phi_n (\vec{r}-\vec{R}+\vec{t}) 
 = \euler^{-\im \vec{k} \vec{t} } \sum_{\vec{R}} \euler^{\im \vec{k}\vec{R} }  \phi_{n}(\vec{r}-\vec{R}), 
\end{equation}
where the second step is due to invariance under a translation of the lattice $\vec{t}$.
The Bloch functions constructed in this way are in general not orthogonal.
By taking a relevant subset of atomic states, and defining for each $\vec{k}$
\begin{equation}
 H_{nm} = \bra{\Psi_{n,\vec{k}}} H \ket{\Psi_{m,\vec{k}}} \text{  and  } S_{nm} = \braket{\Psi_{n,\vec{k}}}{\Psi_{m,\vec{k}}}
\end{equation}
we can set up the secular equation
\begin{equation}
 \det(H_{nm}-E_kS_{nm})=0.
\end{equation}
This gives us the band energies. The corresponding eigenstates are called Löwedin functions.
% BAD writing here, tsss. What is the poinit anyway?
% resemble the Löwedin states the crystal symmetry? They should (--> see Slater et al.)



In this picture we think of electrons as being in a certain state of an atom and hopping to other states rather then being delocalized over the whole crystal.
In any real system the states will have some overlap, creating the possibility for electrons to hop between different sites.

Using second quantization, we can make the above transformation for creation and annihilation operators as well.
The relation between creation and annihilation operators for Bloch states and Wannier states is then given by
\begin{IEEEeqnarray}{rCl}
  a^{\dagger}_{i , n}  =  \sum_{\vec{k}} \euler^{\im \vec{k}\vec{R}_i } a^{\dagger}_{\vec{k} , n}, 
      &\quad&
  a_{i , n}  =  \sum_{\vec{k}} \euler^{-\im \vec{k}\vec{R}_i } a_{\vec{k} , n},  \nonumber\\  
  a^{\dagger}_{\vec{k} , n}  =  \sum_{i} \euler^{-\im \vec{k}\vec{R}_i } a^{\dagger}_{i , n}
    &\quad&
  a_{\vec{k} , n}  =  \sum_{i} \euler^{\im \vec{k}\vec{R}_i } a_{i , n}.
\end{IEEEeqnarray}
By construction they fullfill the anticommutation relations for fermion creation and annihilation operators. 
The correlation is taken into account by introducing a two particle operator. 
Since the single particel Hamiltonian $H_0$ is diagonal for the Bloch states, we can write
\begin{IEEEeqnarray}{rCl}
 H_0 &=& \sum_{\vec{k}} E_{\vec{k}} a^{\dagger}_{\vec{k}} a_{\vec{k}} \nonumber \\
    &=& \sum_{ij} \underbrace{ \frac1N \sum_{\vec{k}} E_{\vec{k}} \euler^{\im \vec{k} ( \vec{R}_i -\vec{R}_j ) } }_{t_{ij}} a^{\dagger}_{i} a_{j},
\end{IEEEeqnarray}
defining the one particle operator $t_{ij}$. 
The diagonal terms $t_{ii}$ are the single particle energys of the atomic sites, 
while the off-diagonal elements provide the hopping.
In cases where the tight binding is a good approximation those hopping elements will fall off quite fast and 
we might include only interactions between close neighbours, usually between nearest neighbours and maybe next-to-nearest neighbours.
So far we did not take electron-electron interactions into account. 
This can be done by introducing the two particle operator
\begin{equation}
 H_{\text{int}} = \sum_{ijkl} U_{ijkl} a^{\dagger}_i a^{\dagger}_j a_k a_l
\end{equation}
where the matrix elements $U_{ijkl}$ are given by
\begin{equation}
 U_{ijkl} = \int \!  \dint^3 r \, \dint^3 r^{\prime} \,  \Psi_i^*(\vec{r}) \Psi_j^*(\vec{r}^{\prime}) V(\vec{r}-\vec{r}^{\prime} ) \Psi_k(\vec{r}) \Psi_l(\vec{r}) 
\end{equation}
Due to the small overlap of different states, usually only a few matrix elements are of relevant magnitude.
The diagonal matrix elements $U_{iii} = U$ account for the repulsion between electrons on the same site. 
Because of the Pauli principle they have to have opposite spins. This is guaranteed by the anticommutation relation of creation and annihilation operators.
Including spin notation in the creation and annihilation operators, we can simplify the interaction contribution to 
\begin{equation}
 H_{\text{int}} = U \sum_i n_{i,\uparrow} n_{i,\downarrow},
\end{equation}
using the number operator $n_{i,\sigma} = a^{\dagger}_{i,\sigma} a_{i,\sigma}$.
This is the only term taken into account for the Hubbard model.
% other interactions 
% U_{ijji}, Heisenberg
% U_{iijj}, density fluctuations

% superexchange?










\subsubsection{Relative Orientation Of Octahedra}

% rotation by 11° in Sr$_2$IrO$_4$, corner sharing
% edge sharing in other materials, non 2D setup



To describe the above spin system there are several models. We assumed the tight binding approximation to be valid, i.e. atomic orbitals
are still a good description for electrons and they are localized. However, their wavefunctions have a certain overlap and there is a correlation. 
%In the Heisenberg model this correlation is not taken into account, the Hamiltonian describes only spin interaction. 
%\begin{equation}
% \hat{H} =  -J \sum_{<i,j>} \hat{\mathbf{S}}_i\hat{\mathbf{S}}_j
%\end{equation}
%where $<i,j>$ indicates a sum over nearest neighbours and the spin operator is $\hat{\mathbf{S}}_i = \frac12 \left( \sigma^x_i, \sigma^y_i, \sigma^z_i \right)^{\intercal}$.
%The model can be extended to next-to-nearest neighbour interactions and so on. 

\section{to be redone}

However, since electrons with opposite spin can occupy the same state and repell each other, this might have to be taken into account, depending on the 
strength of the correlation in such a situation. 
This can be done in the Hubbard model, given by
\begin{equation}
 \hat{H} = \underbrace{U \sum_i c^{\dagger}_{i,\uparrow}c_{i,\uparrow} c^{\dagger}_{i,\downarrow}c_{i,\downarrow} }_{\text{correlation}}
	    -\underbrace{t \sum_{<i,j>,\sigma} c^{\dagger}_{i,\sigma}c_{j,\sigma} + c^{\dagger}_{j,\sigma}c_{i,\sigma} }_{\text{hopping term}}
\end{equation}
with the creation operator 
\begin{equation}
 c^{\dagger}_{i} = \euler^{-\im\pi \sum_{j<i} a^{\dagger}_j a_j } a^{\dagger}_i\quad ; \quad a^{\dagger}_i = \frac12 \left( \sigma^x_i + \im \sigma^y \right)
\end{equation}
and the corresponding annihilation operator. 
 this kind of creation operator phase correction is only needed when derived from the Heisenberg model, see Quantum Many Particle Systems by Negele
In certain situations the Heisenberg model can be derived from the Hubbard model. 

\section{Band Structure}

% Insulator (band splitup using U)
% How U changes the picture


The Hubbard model is defined by the Hamiltonian
\begin{equation}
 \hat{H} = - t \sum_{<i,j>,\sigma} c^{\dagger}_{i,\sigma}c_{j,\sigma} 
	    -\mu \sum_{i,\sigma} c^{\dagger}_{i,\sigma}c_{i;\sigma}
	   + U \sum_i c^{\dagger}_{i,\uparrow}c_{i,\uparrow} c^{\dagger}_{i,\downarrow}c_{i,\downarrow} 
	    . \label{Hubbard_space}
\end{equation}
where the sum runs over all pairs $<i,j>$.
The first term is the band-structure Hamiltonian, while the second term models the electron-electron interaction such that only repulsion on the same site is taken into account. 
U characterizes the strength in this interaction and will be treated as an tunable parameter. 
\todo{digg into the band structure, different examples, expecially the $t-t^{\prime}-t^{\prime\prime}$ model}



In order to get the Hamiltonian in momentum space we make use of the relation between creation and annihilation operators in space and momentum space
\begin{equation}
 c^{\dagger}_i=N^{-\frac12} \sum_k \euler^{-\im \vec{k}\vec{R}_i } c^{\dagger}_k, \qquad c_i=\sum_k N^{-\frac12}\euler^{\im \vec{k}\vec{R}_i } c_k.
\end{equation}
The second term in \ref{Hubbard_space} is the most straightforward one. Since \mbox{$\sum_i \euler^{(\vec{k}-\vec{l})\vec{R}_i } = N\delta_{\vec{k}\vec{l}}$} we
get by simply inserting the above relations
\begin{equation}
 -\mu \sum_{i,\sigma} c^{\dagger}_{i,\sigma}c_{i;\sigma} = 	-\mu \sum_{\vec{k},\sigma} c^{\dagger}_{\vec{k},\sigma}c_{\vec{k}\sigma}
\end{equation}
In a similar manner the first term turns into	
\begin{IEEEeqnarray}{c}
 -\frac{t}{N} \sum_{\vec k \vec l ,\sigma} \sum_{<i,j>} 
	      \left( \euler^{-\im \left(  \vec{k}\vec{R}_i - \vec{l}\vec{R}_j\right)} + \euler^{-\im \left(  \vec{k}\vec{R}_j - \vec{l}\vec{R}_i \right)}\right)    \label{ham_pspace}
\end{IEEEeqnarray}
We can now reparametrize the sum over nearest neighbours, namely
\begin{equation}
 \sum_{<i,j>} = \sum_i \sum_d \quad, \quad \vec{R}_j = \vec{R}_i + \vec{T}_d
\end{equation}
with $\vec{T}_d=R_i-R_j$ for different types of nearest neighbours. In a two dimensional square lattice with lattice constant $a$ we hould have 
$\vec{T}_d \in \{a\cdot\vec{e}_x,a\cdot\vec{e}_y\}$.
We can therefore write \ref{ham_pspace} as
\begin{IEEEeqnarray}{Cl}
 & -\frac{t}{N} \sum_{\vec{k},\vec{l},\sigma} \sum_{i} \euler^{-\im \left(\vec{k}-\vec{l} \right)\vec{R}_i } 
    \sum_d \left(\euler^{-\im \vec{k}\vec{T}_d} + \euler^{\im  \vec{l} \vec{T}_d} \right) 
    c^{\dagger}_{\vec{k},\sigma}c_{\vec{l},\sigma} \nonumber \\
    =& \sum_{\vec{k},\sigma}  c^{\dagger}_{\vec{k},\sigma}c_{\vec{k},\sigma}  \underbrace{\sum_d -2t \cos( \vec{k} \vec{T}_d ) }_{\varepsilon(\vec{k}) }
\end{IEEEeqnarray}
The $U$-term translates to
\begin{IEEEeqnarray}{rl}
 &\frac{U}{N^2} \sum_{\vec{k}\vec{l}\vec{m}\vec{n}} \sum_i \euler^{-\im (\vec{k}-\vec{l}+\vec{m}-\vec{n})\vec{R}_i } 
    c^{\dagger}_{\vec{k},\uparrow}c_{\vec{l},\uparrow} c^{\dagger}_{\vec{m},\downarrow}c_{\vec{n},\downarrow} \nonumber \\
    =& \frac{U}{N} \sum_{\vec{k}\vec{l}\vec{m}\vec{n}} \delta(\vec{k}-\vec{l}+\vec{m}-\vec{n} )
	c^{\dagger}_{\vec{k},\uparrow}c_{\vec{l},\uparrow} c^{\dagger}_{\vec{m},\downarrow}c_{\vec{n},\downarrow} \nonumber \\
    =& \frac{U}{N} \sum_{\vec{k}\vec{k}^{\prime}\vec{q}}
	c^{\dagger}_{\vec{k},\uparrow}c_{\vec{k}-\vec{q},\uparrow} c^{\dagger}_{\vec{k}^{\prime},\downarrow}c_{\vec{k}^{\prime}+\vec{q},\downarrow}
 \end{IEEEeqnarray}
 and the total expression reads now
 \begin{equation}
  \hat{H} = \sum_{\vec{k},\sigma} \left(\varepsilon(\vec{k}) - \mu\right) c^{\dagger}_{\vec{k},\sigma}c_{\vec{k}\sigma} + \frac{U}{N} \sum_{\vec{k}\vec{k}^{\prime}\vec{q}}
	c^{\dagger}_{\vec{k},\uparrow}c_{\vec{k}-\vec{q},\uparrow} c^{\dagger}_{\vec{k}^{\prime},\downarrow}c_{\vec{k}^{\prime}+\vec{q},\downarrow}
 \end{equation}


\section{Mean Field Equations}

We treat the Hubbard modell in a pertubative approach, where the pertubation is given by the Hubbard interaction $H_U$ defined above.
We expect a symmetry breaking ground state \todo{why do we expect SB?}. 
We can describe such a ground state by introducing a off-diagonal propagator in addition to the usual one.
The propagators are defined by
\begin{IEEEeqnarray}{rCl}
 G_{\vec p}(\tau) &=& -\langle \mathcal{T}_{\tau} c_{\vec{p}        ,\sigma}(\tau)  c^{\dagger}_{\vec{p},\sigma}(0) \rangle \\
 F_{\vec p}(\tau) &=& -\langle \mathcal{T}_{\tau} c_{\vec{p}+\vec{Q},\sigma}(\tau)  c^{\dagger}_{\vec{p},\sigma}(0) \rangle \\ \label{Def_Propagator}
\end{IEEEeqnarray}
with the imaginary time ordering operator $\mathcal{T}_{\tau}$, acting on a pair of fermion operators according to
\begin{equation}
 \mathcal{T}_{\tau} \hat{A}(\tau_1) \hat{B}(\tau_2) = -\Theta(\tau_1-\tau_2)\hat{A}(\tau_1) \hat{B}(\tau_2) + \Theta(\tau_2-\tau_1)\hat{B}(\tau_2) \hat{A}(\tau_1).
\end{equation}

where $\vec{Q}=(\pi,\pi)$ is a nesting vector in the Brillouin zone, 
which means that the fermi surface is in great parts symmetric under translations of the nesting vector.
We now calculate the expressions for the propagators in the mean field aproach.
We can rewrite every product of two operators $\hat{A}\cdot\hat{B}$ as
\begin{IEEEeqnarray}{rCl}
 \hat{A}\cdot\hat{B} 
		    &=&	 \left(\hat A - \langle \hat A \rangle \right) \left( \hat B -\langle \hat B \rangle \right)
			 +\langle \hat A \rangle \hat B
			 +\langle \hat B \rangle \hat A
			 - \langle \hat A \rangle \langle \hat B \rangle
\end{IEEEeqnarray}
In the mean field approach we neglect the first factor – the product of fluctuations – leaving us with
\begin{equation}
  \hat{A}\cdot\hat{B} 
		   \stackrel{\mathrm{mf}}{ \approx }
			 \langle \hat A \rangle \hat B
			 +\langle \hat B \rangle \hat A 
			 - \langle \hat A \rangle \langle \hat B \rangle
\end{equation}
We use this relation on the number operators $n_{\sigma}$ in the Hubbard term of the Hamiltonian. 
By dropping the constant term corresponding to $\langle \hat A \rangle \langle \hat B \rangle$, since this doesn't change our system, we end up with
\begin{equation}
 H_U \stackrel{\mathrm{mf}}{\approx}  \frac{U}{N}
 \sum_{\vec{q}} \sum_{\sigma} 
 \left( \sum_{\vec{p}^{\prime}} \langle c^{\dagger}_{\vec{p}^{\prime},-\sigma} c_{\vec{p}^{\prime}+\vec{q},-\sigma} \rangle \right)
	\sum_{\vec p}  c^{\dagger}_{\vec{p},\sigma} c_{\vec{p}-\vec{q},\sigma}
\end{equation}
From our assumption of having only two non-zero propagators we see that we are left with only two non-vanishing contributions from 
$\langle c^{\dagger}_{\vec{p}^{\prime},-\sigma} c_{\vec{p}^{\prime}+\vec{q},-\sigma} \rangle$,
namely for $\vec{q}=0$ and $\vec q = \vec  Q$.
The sum over $\vec q$ reduces then to
\begin{equation}
 H_U \approx \sum_{\sigma} \left( U n_{-\sigma} \sum_{\vec{p}} c^{\dagger}_{\vec{p}, \sigma} c_{\vec p, \sigma} 
	      + \frac{ U m_{s,-\sigma}}2
			      \sum_{\vec p} \left(  c^{\dagger}_{\vec{p}+\vec Q, \sigma} c_{\vec p, \sigma} 
	                                          + c^{\dagger}_{\vec{p}       , \sigma} c_{\vec p+ \vec Q, \sigma} \right) \right)
\end{equation}
For later convenience the last term was added again with the sum  over $\vec{p}$ shifted by $\vec Q$, compensationg with a factor of $\frac12$.
\begin{IEEEeqnarray}{rClLl}
 n_{\sigma} &=&  \frac1{N} \sum_{\vec{p}} \langle c^{\dagger}_{\vec{p},\sigma} c_{\vec{p},\sigma} \rangle %\nonumber \\
	    &=&  \frac1{N} \sum_{\vec{p}} \left( 1- G_{\vec{p},\sigma}(0) \right) \\
m_{s,\sigma} &=& \frac1{N} \sum_{\vec{p}} \langle c^{\dagger}_{\vec{p},\sigma} c_{\vec{p}+\vec{Q},\sigma} \rangle %\nonumber \\
	   &=-&  \frac1{N} \sum_{\vec{p}} F_{\vec{p},\sigma}(0) \label{Def_ms}
\end{IEEEeqnarray}
The coefficient $n_{\sigma}$ counts the amount of up spins, relative to the number of sites $N$, 
that is the filling factor for spin particles taking values in between 0 and 1 and $n_\sigma =\frac12$ for half-filling.
The expression for $m_{s,\sigma}$ is related to the staggered magnetization $m_s$ through
\begin{IEEEeqnarray}{rCl}
 m_s 	&=& \frac1N \sum_{i} \mathrm{sgn}(i) \left( c^{\dagger}_{i,\uparrow}c_{i,\uparrow} - c^{\dagger}_{i,\downarrow}c_{i,\downarrow} \right) \nonumber \\
	&=& m_{s,\uparrow}-m_{s,\downarrow}
\end{IEEEeqnarray}
\todo{more on the meaning of m and n? or later?}
Together we have the total mean-field hamiltonian
\begin{IEEEeqnarray}{rCl}
 H_{\mathrm{mf}} &=
		\sum_{\vec{p},\sigma} &
				      \left( \varepsilon_k - \mu + U n_{-\sigma} \right) 
					  c^{\dagger}_{\vec{p},\sigma} c_{\vec{p},\sigma}
				      +\frac{U}2  m_{s,-\sigma}	 
					  \left( c^{\dagger}_{\vec{p}+\vec{Q},\sigma} c_{\vec{p},\sigma} +\mathrm{h.c.} %c^{\dagger}_{\vec{p},\sigma} c_{\vec{p}+\vec{Q},\sigma} 
					  \right)					 
\end{IEEEeqnarray}
From the definition of the propagtors in \ref{Def_Propagator} we get the equation of motion for them by calculating the imaginary time derivative.
Using the equation of motion for operators, $\frac{\dint}{\dint \tau} \hat{A} = [H,\hat{A}] + \frac{\partial}{\partial_{\tau}} \hat{A}$,  this leads to
\begin{IEEEeqnarray}{rCl}
 \partial_{\tau} G_{\vec{p},\sigma}(\tau) 
 &=&
 \delta(\tau) \langle c_{\vec{p},\sigma}(\tau) c^{\dagger}_{\vec{p},\sigma}(0) + c^{\dagger}_{\vec{p},\sigma}(0) c_{\vec{p},\sigma}(\tau) \rangle \nonumber \\&&
 + \Theta(\tau) \langle [H_{\mathrm{mf}},c_{\vec{p},\sigma}(\tau)] c^{\dagger}_{ \vec{p},\sigma}(0) \rangle		\nonumber \\ &&
 -  \Theta(-\tau) \langle c^{\dagger}_{ \vec{p},\sigma}(0) [H_{\mathrm{mf}},c_{\vec{p},\sigma}(\tau)]  \rangle \label{EOM_G}
\end{IEEEeqnarray}
The commutator can be evaluated using the identity $[AB,C] = A\{B,C\} - \{A,C\}B$
\todo{give fermion commutation rules somewhere earlier}
, leaving us with
\begin{equation}
 [H,c_{\vec{p},\sigma}(\tau)]=-\left(\varepsilon(\vec{p})-\mu + Un_{-\sigma} \right) c_{\vec{p},\sigma}(\tau) - Um_{s,-\sigma} c_{\vec{p}+\vec{Q},\sigma}(\tau)
\end{equation}
Putting this back in \ref{EOM_G} together with the definitions for $G$ and $F$, we get
\begin{IEEEeqnarray}{rCl}
  \partial_{\tau} G_{\vec{p},\sigma}(\tau) 
&=&
\delta(\tau)\langle \{c_{\vec{p},\sigma}(\tau),c^{\dagger}_{\vec{p},\sigma}(0)\} \rangle
- \left( \varepsilon_{\vec{p}}-\mu+ Un_{-\sigma} \right) G_{\vec{p},\sigma}(\tau)  \nonumber \\ &&
 -Um_{s,-\sigma} F_{\vec{p},\sigma}(\tau) \label{EOM_G_II}
\end{IEEEeqnarray}
In a next step we want to take the Fourier transform of this equation. 
The Fourier transformed propagator is related to the propagator in imaginary time through
\begin{IEEEeqnarray}{rCl}
 G_{\vec{p},\sigma}(\tau) &=& \frac{1}{\beta} \sum_n \euler^{-\im \omega_n \tau} G_{\vec{p},\sigma}(\im \omega_n) \\
 G_{\vec{p},\sigma}(\im \omega_n) &=& \int_0^{\beta} \! \!\dint  \tau \: \euler^{\im \omega_n \tau} G_{\vec{p},\sigma}(\tau)
\end{IEEEeqnarray}
The so called Matsubara frequencies $\omega_n$ are given by $\omega_n = \frac{\pi}{\beta}(2n+1), \; n \! \in \! \mathbb{Z}$.
This ensures the anti-periodicity of fermionic Greensfunctions, $G(\tau+\beta) = -G(\tau)$.
Transforming equation \ref{EOM_G_II} to momentum space we get
\begin{IEEEeqnarray}{rCl}
 \left( \im \omega_n -\varepsilon_{\vec{p}} +\mu - Un_{-\sigma} \right) G_{\vec{p},\sigma}(\im \omega_n) = 1 +Um_{s,-\sigma} F_{\vec{p},\sigma}(\im \omega_n)
\end{IEEEeqnarray}
In the same way, starting from $\frac{\dint}{\dint \tau} F_{\vec{p},\sigma}(\tau)$ we get for the off-diagonal propagator
\begin{IEEEeqnarray}{rCl}
 \left( \im \omega_n -\varepsilon_{\vec{p}+\vec{Q}} +\mu - Un_{-\sigma} \right) F_{\vec{p},\sigma}(\im \omega_n) = Um_{s,-\sigma} G_{\vec{p},\sigma}(\im \omega_n)
\end{IEEEeqnarray}
There is no constant term, since the anticommutator in \ref{EOM_G_II} is zero for off-diagonal momenta. 
Putting the last two equations together we get the expressions for the propagators
\begin{IEEEeqnarray}{rCl}
 G_{\vec{p},\sigma}(\im \omega_n) &=& \frac{ ( \im \omega_n - \varepsilon_{\vec{p}+\vec{Q}} + \mu -Un_{-\sigma} )}
			      { ( \im \omega_n - \varepsilon_{\vec{p}+\vec{Q}} + \mu -Un_{-\sigma} )
			        ( \im \omega_n - \varepsilon_{\vec{p}}         + \mu -Un_{-\sigma} )
			      - U^2m_{s,-\sigma}^2 } \nonumber \\
 F_{\vec{p},\sigma}(\im \omega_n) &=& \frac{ Um_{s,-\sigma}}
			    { ( \im \omega_n - \varepsilon_{\vec{p}+\vec{Q}} + \mu -Un_{-\sigma} )
			      ( \im \omega_n - \varepsilon_{\vec{p}}         + \mu -Un_{-\sigma} )
			      - U^2m_{s,-\sigma}^2 }			      
\end{IEEEeqnarray}
We can rewrite this in a more appealing way by factorizing the denominator of both propagators. 
The poles are located at
\begin{equation}
 E_{\vec{p},\sigma}^{\pm}
 =
 \frac{\varepsilon_{\vec{p}}+\varepsilon_{\vec{p}+\vec{Q}}}2 -\mu + Un_{-\sigma}  \pm \sqrt{ \left(\frac{\varepsilon_{\vec{p}}-\varepsilon_{\vec{p}+\vec{Q}}}2\right)^2 + U^2m_{s,-\sigma}^2 }
\end{equation}
Note that $E_{\vec{p},\sigma}^{\pm}=E_{\vec{p}+\vec{Q},\sigma}^{\pm}$, since $\varepsilon_{\vec{p}+2\vec{Q}}=\varepsilon_{\vec{p}}$.
With this, we can write the propagators as
\begin{IEEEeqnarray}{rCl}
 G_{\vec{p},\sigma}(\im \omega_n) &=& \frac{ \im \omega_n - \varepsilon_{\vec{p}+\vec{Q}} +\mu -Un_{-\sigma} }
					    { (\im \omega_n - E_{\vec{p},\sigma}^+) (\im \omega_n - E_{\vec{p},\sigma}^-) }
\\
 F_{\vec{p},\sigma}(\im \omega_n) &=& \frac{ Um_{s,-\sigma} }
					    { (\im \omega_n - E_{\vec{p},\sigma}^+) (\im \omega_n - E_{\vec{p},\sigma}^-)}
\end{IEEEeqnarray}


%\begin{IEEEeqnarray}{rCl}
%  G_{\vec{p},\sigma}(\im \omega_n) &=& \frac{ u_{\vec{p}        ,\sigma }}{ (\im \omega_n - E_{\vec{p},\sigma}^+) }
%			+ \frac{ u_{\vec{p}+\vec{Q},\sigma }}{ (\im \omega_n - E_{\vec{p},\sigma}^-) } \\
%  F_{\vec{p},\sigma}(\im \omega_n) &=& \frac{ \tilde{u}_{\vec{p},\sigma }}{ (\im \omega_n - E_{\vec{p},\sigma}^+) }
%			- \frac{ \tilde{u}_{\vec{p},\sigma }}{ (\im \omega_n - E_{\vec{p},\sigma}^-) }
%\end{IEEEeqnarray}
%where $u$ and $\tilde{u}$ are given by
% \begin{IEEEeqnarray}{rCl}
% u_{\vec{p},\sigma} &=& \frac{E_{\vec{p},\sigma}^+ - \varepsilon_{\vec{p}+\vec{Q}} +\mu -Un_{-\sigma} }{E_{\vec{p},\sigma}^+ -E_{\vec{p},\sigma}^-} \\
% \tilde{u}_{\vec{p},\sigma} &=& \frac{Um_{s,-\sigma}}{E_{\vec{p},\sigma}^+ -E_{\vec{p},\sigma}^-}.
% \end{IEEEeqnarray}
% \todo{connection to diagrammatic approach, resolve $1-n_{-\sigma} \stackrel{?}{=} n_{-\sigma}$ issue.} 
% \todo{ How to draw nice diagrams in \LaTeX?}




\subsection{Half-Filling}

The materials investigated in this work are half-filled and have a spin independent Hamiltonian.
Therefore we can assume symmetry between up and down states, and get as a result
$n=n_{\uparrow}+n_{\downarrow}=2\cdot n_{\uparrow}= 2\cdot n_{\downarrow} = 1$.

This corresponds to a chemical potential of 
\todo{explain the connection between half filling and $U$}
$\mu=\frac{U}2$.
We furthermore assume that $m_{s,\uparrow}=-m_{s,\downarrow}$. 
The staggered magnetization may thus be expressed by $m_s=2\sigma \cdot m_{s,\sigma}$.
This relation does not hold in general, since double occupied states give a symmetric contribution to $m_{s,\sigma}$ rather than a asymmetric one.
However, in an otherwise asymetric large system these contributions are expected to vanish, since they might be distributed evenly on even and odd sites 
and therefore cancel on average.

In order to calculate $m_ {s,\sigma}$ we solve equation \ref{Def_ms}, where we can replace $m_{s,\uparrow}$ by $-m_{s,\downarrow}$, as stated above.
We furthermore drop the spin labels on $E_{\vec p,\sigma}^{\pm}$, since they depend only on $m_{\sigma}^2$.  
This decouples the two equations for $\uparrow$ and $\downarrow$ and by using the definition of the off-diagonal propagator $F_{\sigma,\vec{p}}$ we end up with
\begin{equation}
 m_{s,\sigma} = \frac1N \sum_{\vec{p}} \sum_{n\in \mathbb{Z}} 
							      \frac { Um_{s,\sigma} }
								    { (\im \omega_n - E_{\vec{p}}^+) (\im \omega_n - E_{\vec{p}})}
\end{equation}
The $m_{s,\sigma}$ on the LHS is canceled by the one in the RHS, but $E^{\pm}$ is still dependent on $m_{s,\sigma}$.
To deal with the summation over the Matsubara frequencies $\im \omega_n$, we use Cauchys integral theorem as descriped in \ref{MFS}.
The resulting equation
\begin{equation}
 1= \frac{U}{N}\sum_{\vec{p}} \frac{1}{E_{\vec p}^+ - E_{\vec{p}}^-} \left( \frac{1}{1+\euler^{\beta E_{\vec p}^+}} - \frac{1}{1+\euler^{\beta E_{\vec p}^-}} \right)
\end{equation}
is solved using the Newton-Raphson method.
In the case of a square lattice with only nearest neighbour hopping terms the staggered magnetization as a function of the dimensionless parameter $\frac{U}{t}$ is shown in figure
\ref{ms_nn}.

\begin{figure}
 \label{ms_nn}
 \includegraphics[width=.9\textwidth]{../stagmag_T10.png}
 \caption{staggered magnetization as function of $\frac Ut$}
 
\end{figure}



We get an expression for $n_{\sigma}$ in a similiar way summing up $G_{\vec p,\sigma}$. 
Since the number of particles is fixed by the chemical potential, and it can be shown that half filling, $n_{\uparrow}=n_{\downarrow}=\frac12$ occurs at $\mu=\frac U2$ 
this has only been done to show self consistency.

\section{Calculating the magnetic response function}

diagrams, approximations: non-crossing, restricting to ladder diagrams (using the mean-field props), summing up, using matrices and geometric sum.
Retarded response function, Wick rotation. 
$\chi^{+-},\chi^{zz}$
evaluating infinite frequency sums. 

$A=\Im(C^+)$


\section{Results}
Temperature: atm T=10K. Why, what changes?



shows dispersion along the zone boundary,
difference in (pi,0) compared to (pi/2,pi/2)
compare to heisenberg, show limit
continuum


\subsection{t-U-Model}
compare to measurements
\subsection{t-t'-t''-U-Model}
explain wher t-t'-t'' is from. reference
show energy dispersion
explain features from that (e.g. form of continuum boundary)
compare to measurements: extract measurement data, choose same path, set atop each other, find perfect U. Interpretation? Continuum?
compare to J-J'-J'' Heisenberg. Can this explain those features? Advantage of this method

broadness of peaks, smearing through $\eta$, getting a better continuum?
\newpage

TO DO
\begin{itemize}
 \item $T=0$;
 \item Results, inkl. plot
\end{itemize}









\appendix

\chapter{Mathematical techniques}

\section{Matsubara Frequency Summation} \label{MFS}

In the calculations above we encounter frequently functions of the type 
$ \sum_{n \in \mathbb{Z}} f(\im \omega_n)$.

In order to deal with these infinite sums over Matsubara frequencies, we make use of Cauchys integral theorem twice.
Fermionic Matsubara frequencies are given by $\frac{\pi}{\beta}(2n+1)$ for $n \in \mathbb{Z}$.
Those are also the roots of $h(z) = \frac{\beta}{\euler^{\beta z}+1}$.
If $f(z)$ doesn't have any poles on the imaginary axis, we can therefore rewrite the sum as an integral, 
with an contour that encloses the imaginary axis tightly and cloeses at infinity.\todo{as shown in figure}
The functions used here usually have poles on the real axis. 
With choosing the contour close enough to the imaginary axis we keep those outside of the enclosed area.
For functions falling off faster than $\frac{1}{|z|^{1+\delta}}$ $(\delta>0)$ as $|z|\rightarrow \infty$ we can blow up the contour to a
circle with infinite radius. The integral will vanish, but we pick up residuals for every pole of $f$.
In total we get
\begin{equation}
 \sum_{n\in\mathbb{Z}} f(\im \omega_n) = \frac{1}{2\pi \im} \int_0 \dint z f(z)\frac{\beta}{1+\euler^{\beta z }} 
 = \sum_{\mathrm{Res}_f} \left. \frac{f(z)\beta}{1+\euler^{\beta z }}\right|_{z=z_{\mathrm{Res}_f}}
\end{equation}

\section{Wick Rotation}




\bibliographystyle{plain}
\bibliography{masterthesis}

\end{document}
