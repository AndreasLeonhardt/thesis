\documentclass[a4paper,10pt]{report}
\usepackage[utf8]{inputenc}
\input{/home/andi/Studium/LaTex/preambel.tex}


%opening
\title{Magnetic Excitations In Iridiates}
\author{Andreas Karl Leonhardt}

\begin{document}

%\maketitle

%\begin{abstract}

%\end{abstract}
\tableofcontents

\chapter{Introduction}
\section{Towards A Spin Hamiltonian}

\subsection{Iridium And Compounds}
Iridium is a transition metal, i.e. it has a only partially filled $d$ shell, the shell structure of atomar Ir is given by $[\mathrm{Xe}]4f^{14}5d^7 6s^2$.
% ${\mathrm{Sr}}_{2}{\mathrm{IrO}}_{4}$
In the compounds treated in this thesis Iridium is oxidized to $\mathrm{Ir}^{4+}$ 
removing the $6s^2$ electrons as well as two electrons from the $5d$ shell. 
the ion has therefore $5d^5$ as the outermost and active shell ~\cite{Abragam70}.

%(the ion shell structure does not follow the Aufbau principle. 
%Therefore has $\mathrm{Ir}^{4+}$ a different shell structure than Tantalum, even though they have the same number of electrons. 
%This is because of the different charge of the nucleus, that leads to a greater seperation in energs levels with different quantum number n ($\propto \frac{Z}{n^2}$)).

The other elements of the compunds treated in this thesis have closed shells an are therefore spin free. In the case of $\mathrm{Ir}^{4+}$, 
$\mathrm{O}^{2-}$ has only closed shells ($1s^2,2s^2,2p^6$) and therefore is $L=S=0$.
the same holds for $\mathrm{Sr}^{2+}$ with the electron configuration of $\mathrm{Kr}$

\subsection{Ligand Field}

Embedded in the crystal of the compound, total rotational symmetry is broken and reduces to discrete symmetries about certain axes. 
This splits up the former degenerate $5d$ levels. 
In all of the compunds lookes at here, the $\mathrm{Ir}$ atoms are surrounded by octahedral of double negatively charged oxygen ions, the so calles ligands. 
Those octahedra might share corners or even edges, while the latter one simply means they share two edges. 
%
% In ${\mathrm{Sr}}_{2}{\mathrm{IrO}}_{4}$ the $\mathrm{Ir}^{4+}$ is surrounded by 8 $\mathrm{O}^{2-}$, building a octaeder with shared corners.
%
The field generated by those ligands split up the $5d$ levels in the threefold $t_{2g}$ and the twofold $e_g$ states. 
Since the ligand field is quite strong
\todo{quote, strong compared to what, Hund's rule and why it is violated}
only the $t_{2g}$ levels are populated in the ground state, leaving one hole in this band. 
The $e_g$ orbitals are not occupied in the ground state, since the crystal field splitting is stronger than the repulsion of electrons beeing in the same orbital. 
This leads to the so calles low-spin state, in contrast to a weak crystal field, 
where it would be more favourable to follow Hund's rules, 
i.e. filling up the $t_{2g}$ and the $e_g$ orbitals with electrons of the same spin, before putting to electrons with opposite spin into an orbital.
The $t_{2g}$ levels build an \emph{effective} $L=1$ system, since it is threefold degenerate.

\subsection{Strong Spin-Orbit Coupling}

In the $5d$ elements spin orbit coupling (SOC) plays an important role since it increases with the charge of the nucleus.
\todo{theoretical background for string SOC according to Z.}
Since there is one hole in the $t_{2g}$ shell left, the total Spin is $S=\frac12$. 
When coupled to the effective spin $L=1$ as mentioned above, 
the sixfold degenerate $t_{2g}$ states (including spin degeneracy) split into a fourfold degenereate level with $J=\frac32$
and twofold degenerate level with $J=\frac12$. The latter one is higher in energy, since we couple an \emph{effective} angular momentum to the spin,
thus the paralell coupling is energetically favourable.
In the ground state the $J=\frac32$ band will be filled, while the $J=\frac12$ band is half filled, leaving us with an effective Spin$\frac12$ system on each Iridium site in the grid.
\todo{create/include graph similar to \cite{PhysRevLett.101.076402}}

The two states are given by a linear combination of the molecular orbits and spin states,
\begin{equation}
 \ket{J_{\mathrm{eff}} =\frac12, M_{J_{\mathrm{eff}}}= \pm \frac12}
 = 
 \frac{1}{\sqrt{3}} \left( \ket{yz,\pm \sigma} \mp \im \ket{zx,\pm \sigma} \mp \ket{xy,\mp \sigma} \right).
\end{equation}
where $+\sigma = \uparrow, -\sigma = \downarrow$. The moelcular orbits are given by linear combinations of the spherical harmonics
\begin{IEEEeqnarray}{rCl}
\ket{xz} &=& \frac{1}{\sqrt{2}} \left( Y_2^{-1} - Y_2^{1} \right) \\
\ket{yz} &=& \frac{\im}{\sqrt{2}} \left( Y_2^{-1} + Y_2^{1} \right) \\
\ket{xy} &=& \frac{\im}{\sqrt{2}} \left( Y_2^{-2} - Y_2^{2} \right) 
\end{IEEEeqnarray}

\todo{transformation properties of the ground state}


\subsection{Spin System And The Hubbard Model}

\subsubsection{Tight Binding Model} % in second quantization

The crystal potential consists of the sum over atomic potentials. 
If the overlap of those is small, we might use atomic states as a starting point for our description of the system.
In this picture we think of electrons as being in a certain state of an atom and hopping to other states rather then being delocalized over the whole crystal.
Energy eigenstates of the crystal must be Bloch states, defined by  the relation
\begin{equation}
 \Psi_k \left( \vec{r}+\vec{R}\right) = \euler^{\im \vec{R} \vec{k} } u_k \left( \vec{r} \right),
\end{equation}
where $\vec{R}$ denotes a translation vector of the lattice and $u_k$ is a periodic function with the same periodicy than the lattice.
In the case of infinitly seperated atoms $u_k$ approaches the sum of atomic orbitals on the different sites of the atoms. 




\subsubsection{Relative Orientation Of Octahedra}

% rotation by 11° in Sr$_2$IrO$_4$, corner sharing
% edge sharing in other materials



To describe the above spin system there are several models. We assumed the tight binding approxiamtion to be valid, i.e. atomic orbitals
are still a good description for electrons and they are localized. However, their wavefunctions have a certain overlap and there is a correlation. 
%In the Heisenberg model this correlation is not taken into account, the Hamiltonian describes only spin interaction. 
%\begin{equation}
% \hat{H} =  -J \sum_{<i,j>} \hat{\mathbf{S}}_i\hat{\mathbf{S}}_j
%\end{equation}
%where $<i,j>$ indicates a sum over nearest neighbours and the spin operator is $\hat{\mathbf{S}}_i = \frac12 \left( \sigma^x_i, \sigma^y_i, \sigma^z_i \right)^{\intercal}$.
%The model can be extended to next-to-nearest neighbour interactions and so on. 



However, since electrons with opposite spin can occupy the same state and repell each other, this might have to be taken into account, depending on the 
strength of the correlation in such a situation. 
This can be done in the Hubbard model, given by
\begin{equation}
 \hat{H} = \underbrace{U \sum_i c^{\dagger}_{i,\uparrow}c_{i,\uparrow} c^{\dagger}_{i,\downarrow}c_{i,\downarrow} }_{\text{correlation}}
	    -\underbrace{t \sum_{<i,j>,\sigma} c^{\dagger}_{i,\sigma}c_{j,\sigma} + c^{\dagger}_{j,\sigma}c_{i,\sigma} }_{\text{hopping term}}
\end{equation}
with the creation operator 
\begin{equation}
 c^{\dagger}_{i} = \euler^{-\im\pi \sum_{j<i} a^{\dagger}_j a_j } a^{\dagger}_i\quad ; \quad a^{\dagger}_i = \frac12 \left( \sigma^x_i + \im \sigma^y \right)
\end{equation}
and the corresponding annihilation operator. 
In certain situations the Heisenberg model can be derived from the Hubbard model. 

\section{Band Structure}

% Insulator (band splitup using U)
% How U changes the picture



\bibliographystyle{plain}
\bibliography{masterthesis}

\end{document}
