\chapter{Introduction}

% remember: there will be an abstract, that contains some introdcutory blabla.
Iridates is the name for a group of anorganical chemical compounds that contain oxydized Iridium.

% interesting examples here
These materials gained a lot of attention in the last decade due to their variety of interesting configurations.
Due to their configuration of an Iridium ion surround by several oxygen anions, they can be described as effective spin systems.
Therefore they are a realization of interesting theretical models such as the Hubbard and the Heisenberg model.
Some of the compounds have a layered structure, leading to low-dimensional systems.


Iridates share a lot of properties with the cuprates. 
Ever since high temperature superconductivity (HTSC) was discovered in cuprates, 
they were intensively investigated. 
Due to the chemical similarity of irdium to copper, it is possible to create materials that share the same geometry as the ones known from cuprates.
Replacing copper with irdium yields interesting effects due to their physical differneces.
The higher charge of the iridium nucleus for example enhances the spin-orbit coupling in the active orbitals.
This creates interesting new effects when embedded in a pervoskite structure, as it is the case in the iridates.
%
% same shit, different try
%
Iridium and Copper have similar chemical properties. 
Iridates are therefore quite similar to Cuprates. 
The later one are intensively investigated ever since high temperature superconductivity was first realized in doped cuprates in 1986 \cite{}.
Both are strongly interacting, which makes them a Mott-insulator at half filling.
Since Iridium is a heavier element some of the parameters are different for Iridates. 
Due to the larger charge of the nucleus, spin-orbit coupling (SOC) is enhanced in Iridates compared to cuprates.
This creates new effects and changes the macroscopic behaviour. 
We find Iridates that behave like soin$\frac12$ systems with half filled bands. 


%They are promising candidates to provide a realization of theoretical models.

The term was coined in resemblance of cuprates, a similar group of materials with Copper instead of Iridium.
% However, both terms are not strictly defined and may include some exceptions with non-ionic bounds.
Cuprates are intensively investigated since the 1980s, when high temperature superconductivity was first realized in Cuprates.


Here we refer to a certain subclass of these materials, where ions of Iridium and Copper respectively are embedded in an octahedron of oxygen ions.
Depending on their geometries, these materials show a wide range of interesting effects.
They provide a realization of two-dimensional spin systems with strong interactions. 

% some words on cuprates and high temperature superconductivity
% similarity to Iridates
% differences in Iridates
% first experimental realization of Iridates. 
% Further specialization on materials investigated here.

% it is insulating, unexpected, compare to cuprates


We find realizations of low-dimensional spin systems

The research on Iridates, both experimentally and theoretically, became an area of intesiv research in the recent years. 
Their share a lot of features with the Cuprates, since Iridium and Copper are almost identical in their chemical properties. 

However, since Iridium is a heavier element, its compounds show new features that differ from the ones known from cuprates.
The most interesting effect is the enhanced spin-orbit coupling due to the heavier nucleus.
By this, the replacement of Copper by Iridium might have drastic on the macroscopic behaviour. 
One example is a transition from the conductor to an Mott-insulator in Sr$_2$XO$_4$, 

Depending on the orientation of these octahedra, the number of oxygen ions shared between two octahedra varies for different materials. 





\section{Motivation}

% Why are Iridates interesting
% - similar to Cuprates, and they pffer a variety of materials with interesting properties, such as HTSC. 
% - - similar chemical configuration, but higher Z lead to :
% - - stronger Spin-Orbit coupling might introduce new effects. (\lambda \propto Z^4)
% - geometric structure: layered and therefore effective 2D systems (pervoskite)
% -- pervoskite structure and SOC create effective spin-$\frac12$ systems
% -- different geometries of spin sites: rectangular, triangular, realization of Kitaev Model.
% -outline doping?
% theoretical point of view: Playground for Hubbard model


% approach outline: 
% -Describe as spin system
% -treat in mean field approach (taken from cuprates)
% -calculate observables and compare to experiment for specific material(s)
% -determine parameters of model (basically U)
% - show that this is a reasonable approach worth for further usage.
% -compare models: Hubbard vs. Heisenberg, 
%  -or since Hubbard $\stackrel{U\rightarrow }infty}{\longrightarrow}$ Heisenberg, is Heisenberg sufficient, valid, good?
% - find ground state: Mott insulator vs … ? 

% What could be done with that? 
% engineering of Kitaev model & other stuff?
% HTSC for doped Iridates?

% controversy about:
% -Mott insulator (paper) (see above)
% -parameters, especially U
% -if HTSC is possible
% need better unterstanding in general blabla.



\section{Outline}

First the characteristic configuration of the crystal is  investigated.
We will see, how Iridium embedded in an octahedron of oxygen ions  with strong  SOC
creates states with an effective total angular momentum of $J=\frac32$ and $J=\frac12$.
While all $J=\frac32$- states are occupied, the later ones are half filled in the undoped case.
They form the active band, which allows us to proceed to a pure spin model using the tight binding approximation.
We introduce  correlations through on-site repulsions. 
This so called Hubbard interaction is the simplest way of accounting for interactions 
and has been shown to provide a good description in similar systems, e.g. the cuprates.
% $t-U$ and $t-t^{\prime}-t^{\prime \prime}-U$
% octa

We then solve the Hubbard model in the mean field approach.
The goal is to calculate the dynamical magnetic susceptibility.
From this we can extract the spin wave dispersion, which 
can be directly compared to results from neutron and X-Ray scattering experiments.
We follow the calculation scheme by  \citet{PhysRevB.65.132404} that was used for the structurally similar cuprate La$_2$CuO$_4$.

The calculations will first be carried out in the large U limit and compared to the results of linear spin waves in a Heisenberg model.
Since the Heisenberg model can be dervied from the Hubbard model in this limit, we can use this to validate the calculations.

We then calculate the dispersion with parameters from Sr$_2$IrO$_4$ for different band structures.
The results were compared to measurements on this material. 
The interaction parameter $U$ is the only adjustable parameter in this approach and will be fitted to give the best match to observations.
We show that the approach outlined above provides results that agree well with measurements and is capable of reproducing the main features of the dispersion.
%
% necessity of t-t'-t''?


