\chapter{Introduction}





The iridates is a large group of materials, that became an area of active research, both experimentally and theoretically. 
The term iridates comprises a large group of materials that contain an anionic complex 
consisting of an iridium cation, which gives the group its name, 
and several negatively charged anions, the so called ligands.
The oxidation level of iridium and the number of ligands per iridium ion depend on the specific component.



% interesting examples here
These materials provide a rich variety of configurations and realizations of interesting physical effects.
The crystal structure includes effectively two dimensional layered  and bi-layered systems, 
honeycomb lattices and more complicated three dimensional structures. 

The same structures can be found in cuprates. 
The cuprates form a class of materials equivalent to the iridates, where Ir is replaced by Cu. 
Chemically these two elements behave similar.
They have the same type of active orbitals, but from different shells. 
This results in slightly changed parameters. 
Otherwise the same types of compounds as in the iridates can be found. 
Cuprates are best known for their high temperature superconductivity (HTSC).
The discovery of HTSC in cuprates 1986 \cite{cuprateHTSC} 
started a still on-going era of intensive research on cuprates and their theoretical description through effective models.
The most important ones  among those  are and the Hubbard model and spin models such as the Heisenberg model, whereas the latter one can 
derived from the Hubbard model for a large interaction strength. 


The electrons in iridates are correlated due to strong interactions.
This leads to new physical effects such as long range ordering,
which itself influences magnetic and electrical properties,
such as anti-ferromagnetic ground states and changes in the conductivity. 

Even though Ir and Cu behave almost identical in terms of chemical bonding, 
their difference in the atomic number changes some parameters that influence the dynamics of the crystal. 
Firstly, Ir is bigger in the sense, that its orbitals are spatially more extended. 
This alters the crystal structure in some of the materials, but more importantly enhances spin orbit coupling.
Stronger SOC is the main difference between iridates and cuprates and leads to new physical phenomenon. 

Iridiates have been synthesized as early as 1969 \cite{Longo1969191}, 
investigations on its structure started in the 90's and intensified 
even more after 2009, when resonant inelastic X-ray scattering (RIXS) was used to 
reveal the structure of cuprates and similar compounds \ref{RIXS}. 

%proposed HTSC, dispute

% most important examples of iridates
The most important examples of iridates 
are the effectively two dimensional single and double layered perovskites Sr$_2$IrO$_4$ and Sr$_3$Ir$_2$O$_7$, which are insulating.
Together with the metallic SrIrO$_3$ they form the Ruddlesden-Popper series.
There are more components of the same structure, which use another type of rare earth metal instead of strontium. 
Another famous iridate configuration is the honeycomb lattice, which is for example found in Na$_2$IrO$_3$ and Li$_2$IrO$_3$.




%
% some words on cuprates and high temperature superconductivity
% similarity to Iridates
% differences in Iridates
% first experimental realization of Iridates. 
% Further specialization on materials investigated here.

% it is insulating, unexpected, compare to cuprates


% low-dimensional spin systems
 






\section{Motivation and Goals}


We will focus in this thesis on the iridate compound Sr$_2$IrO$_4$. 
Despite intensive research in the last years, 
there are still a lot of uncertainty about the exact mechanisms that result in the properties we observe.

Sr$_2$IrO$_4$ is an insulator, which can not be explained from band structure alone. 
Possible mechanisms are of the Mott or Slater type \cite{PhysRevB.89.165115}.
They depend on the strength of SOC and inter-electronic interaction.
Especially the later one is not directly observable and is often only estimated.
We use the Hubbard interaction, which is the simplest way of introducing correlations.
Its only parameter is the on-site interaction $U$, which we will determine in this work.


Another interesting observation in Sr$_2$IrO$_4$ is a weak ferromagnetic moment.
The on-site magnetization for this moment is an order of magnitude smaller than what one would expect from atomic states.
Furthermore shows the material no ferromagnetic but rather anti-ferromagnetic order.
Due to its structure, the ferromagnetic moment can be related to the order parameter of the anti-ferromagnetic ground state.
% does the last sentence make any sense?

The cuprate La$_2$CuO$_4$ is a high temperature superconductor with the same structure as Sr$_2$IrO$_4$. 
Often one chooses a pure spin model to describe their dynamics. 
A large repulsive interaction justifies this at half filling.
In iridates the particle interaction is smaller and the usage of a spin model is not that well justified.
We use the Hubbard model based on the band structure  instead. 
We want to show, that the Hubbard model solved at the mean field level is capable of providing a good description of the spin excitations.






% Why are Iridates interesting
% - similar to Cuprates, and they pffer a variety of materials with interesting properties, such as HTSC. 
% - - similar chemical configuration, but higher Z lead to :
% - - stronger Spin-Orbit coupling might introduce new effects. (\lambda \propto Z^4)
% - geometric structure: layered and therefore effective 2D systems (pervoskite)
% -- pervoskite structure and SOC create effective spin-$\frac12$ systems
% -- different geometries of spin sites: rectangular, triangular, realization of Kitaev Model.
% -outline doping?
% theoretical point of view: Playground for Hubbard model


% approach outline: 
% -Describe as spin system
% -treat in mean field approach (taken from cuprates)
% -calculate observables and compare to experiment for specific material(s)
% -determine parameters of model (basically U)
% - show that this is a reasonable approach worth for further usage.
% -compare models: Hubbard vs. Heisenberg, 
%  -or since Hubbard $\stackrel{U\rightarrow }infty}{\longrightarrow}$ Heisenberg, is Heisenberg sufficient, valid, good?
% - find ground state: Mott insulator vs … ? 

% What could be done with that? 
% engineering of Kitaev model & other stuff?
% HTSC for doped Iridates?

% controversy about:
% -Mott insulator (paper) (see above)
% -parameters, especially U
% -if HTSC is possible
% need better unterstanding in general blabla.



\section{Outline}
In the first part of the thesis we will derive the Hubbard model as an effective model for the iridates. 
First we describe how effective $J= \frac12$ and $J=\frac32$ states emerge from the interplay of two 
effects, the field of the negatively charged oxygen ions and SOC. 


While all $J=\frac32$- states are occupied, the later ones are half filled in the undoped case.
They form the active band, which allows us to proceed to a pure spin model using the tight binding approximation.
We introduce  correlations through on-site repulsions. 
This so called Hubbard interaction is the simplest way of accounting for interactions 
and has been shown to provide a good description in similar systems, e.g. the cuprates.
% $t-U$ and $t-t^{\prime}-t^{\prime \prime}-U$
% octa

We then solve the Hubbard model in the mean field approach.
The goal is to calculate the dynamical magnetic susceptibility.
From this we can extract the spin wave dispersion, which 
can be directly compared to results from neutron and X-Ray scattering experiments.
We follow the calculation scheme by  \citet{PhysRevB.65.132404} that was used for the structurally similar cuprate La$_2$CuO$_4$.

The calculations will first be carried out in the large U limit and compared to the results of linear spin waves in a Heisenberg model.
Since the Heisenberg model can be dervied from the Hubbard model in this limit, we can use this to validate the calculations.

We then calculate the dispersion with parameters from Sr$_2$IrO$_4$ for different band structures.
The results were compared to measurements on this material. 
The interaction parameter $U$ is the only adjustable parameter in this approach and will be fitted to give the best match to observations.
We show that the approach outlined above provides results that agree well with measurements and is capable of reproducing the main features of the dispersion.
%
% necessity of t-t'-t''?


