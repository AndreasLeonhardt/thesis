\chapter{Introduction}


Transition metal oxides (TMOs) build a class of fascinating materials, whose elements 
show a  variety of  condensed matter systems and with multifaceted properties.
Various different structures  can be realized  by choosing the right combination of elements, doping and different preparation techniques.
This gives not only the possibility to find practical realizations of theoretical  models like low-dimensional systems and spin models,
but leads also to the finding of new effects in condensed matter systems.
The resulting materials range from metals to insulators, with all sorts of magnetic properties. 
% This allows to create systems and controlling with different parameter sets and rich 
% Properties like their crystal structure 





The most famous group among them are the cuprates. 
Their defining component is an anionic complex containing oxidised copper. 
The discovery of high temperature super conductivity in some of its compounds in 1986 \cite{cuprateHTSC} 
started an era of intensive research on the cuprates that lasts until today.

In the search for interesting physics new classes were created by replacing some of the elements in the compounds, 
expanding thereby parameter space of defining basic effects to new magnitudes  and combinations.
The iridates are such a group, where copper in the anionic component is replaced by iridium. 
While the first components have been synthesized as early as 1969 \cite{Longo1969191},
research  on the iridates, both experimentally and theoretically, has intensified greatly in the last decade.
Especially since new experimental techniques like resonant  inelastic X-ray scattering were discovered, that help to unveil their inner structure.








%
% some words on cuprates and high temperature superconductivity
% similarity to Iridates
% differences in Iridates
% first experimental realization of Iridates. 
% Further specialization on materials investigated here.

% it is insulating, unexpected, compare to cuprates


% low-dimensional spin systems
 






\section{Motivation and Goals}


We will focus in this thesis on the iridate compound Sr$_2$IrO$_4$. 
Despite intensive research in the last years, 
there is still ongoing dispute about the exact mechanisms that result in some of its properties.


Sr$_2$IrO$_4$ is an insulator, which can not be explained from band structure alone. 
Possible mechanisms are of the Mott or Slater type \cite{PhysRevB.89.165115}.
They depend on the strength of SOC and inter-electronic interaction.
Especially the strenght of the interaction is not directly observable and is often only estimated.
We use the Hubbard interaction, which is the simplest way of introducing correlations.
Its only parameter is the on-site interaction $U$, which we will determine in this work.


Another interesting property of Sr$_2$IrO$_4$ is its  weak ferromagnetic moment.
The on-site magnetization for this moment is an order of magnitude smaller than what one would expect from atomic states.
Furthermore shows the material no ferromagnetic but rather anti-ferromagnetic ordering in the ground state.
Due to its canted crystal structure, the ferromagnetic moment can be related to the order parameter of the anti-ferromagnetic ground state.
We will determine the order parameter in the ground state, which we can then relate to the measured magnetic moment.
% does the last sentence make any sense?

The cuprate La$_2$CuO$_4$ is a high temperature superconductor with the same structure as Sr$_2$IrO$_4$. 
Often one chooses a pure spin model to describe their dynamics. 
A large repulsive interaction justifies this at half filling.
In iridates the particle interaction is smaller and the usage of a spin model is not that well justified.
We use the Hubbard model based on the band structure  instead. 
We want to show, that the Hubbard model solved at the mean field level is capable of providing a better description of the spin excitations.
\Sriro might show superconductivity when doped away from half filling. Understanding the mechanisms defining its basic magnetic excitations might help to find an answer to that question.






% Why are Iridates interesting
% - similar to Cuprates, and they pffer a variety of materials with interesting properties, such as HTSC. 
% - - similar chemical configuration, but higher Z lead to :
% - - stronger Spin-Orbit coupling might introduce new effects. (\lambda \propto Z^4)
% - geometric structure: layered and therefore effective 2D systems (pervoskite)
% -- pervoskite structure and SOC create effective spin-$\frac12$ systems
% -- different geometries of spin sites: rectangular, triangular, realization of Kitaev Model.
% -outline doping?
% theoretical point of view: Playground for Hubbard model


% approach outline: 
% -Describe as spin system
% -treat in mean field approach (taken from cuprates)
% -calculate observables and compare to experiment for specific material(s)
% -determine parameters of model (basically U)
% - show that this is a reasonable approach worth for further usage.
% -compare models: Hubbard vs. Heisenberg, 
%  -or since Hubbard $\stackrel{U\rightarrow }infty}{\longrightarrow}$ Heisenberg, is Heisenberg sufficient, valid, good?
% - find ground state: Mott insulator vs … ? 

% What could be done with that? 
% engineering of Kitaev model & other stuff?
% HTSC for doped Iridates?

% controversy about:
% -Mott insulator (paper) (see above)
% -parameters, especially U
% -if HTSC is possible
% need better unterstanding in general blabla.



\section{Outline}
In the first part of the thesis we will derive the Hubbard model as an effective model for the iridates, starting from its components and the crystal structure. 
First we describe how effective $J= \frac12$ and $J=\frac32$ states emerge from the interplay of two 
effects, the field of the negatively charged oxygen ions and SOC. 


We show then how we can create an effective model for the half filled spin-$\frac12$ band in the tight binding approximation. 
We introduce then  correlations between electrons, modelled as on-site repulsions. 
This so called Hubbard interaction is the simplest way of accounting for interactions 
and has been shown to provide a good description in similar systems, e.g. the cuprates.
% $t-U$ and $t-t^{\prime}-t^{\prime \prime}-U$
% octa

We then solve the Hubbard model in the mean field approach.
The goal is to calculate the dynamical magnetic susceptibility.
From this we can extract the spin wave dispersion, which 
can be directly compared to results from neutron and X-Ray scattering experiments.
We follow the calculation scheme by  \citet{PhysRevB.65.132404} that was used for the structurally similar cuprate La$_2$CuO$_4$.

The calculations will first be carried out in the large U limit and compared to the results of linear spin waves in a Heisenberg model.
Since the Heisenberg model can be dervied from the Hubbard model in this limit, we can use this to validate the calculations.

We then calculate the dispersion with parameters from Sr$_2$IrO$_4$ for different band structures.
The results were compared to measurements on this material. 
The interaction parameter $U$ is the only adjustable parameter in this approach and will be fitted to give the best match to observations.
We show that the approach outlined above provides results that agree well with measurements and is capable of reproducing the main features of the dispersion.
%
% necessity of t-t'-t''?


