
 
\chapter{Results}

\section{Specifications} %of the Calculation}

For all calculations we use dimensionless quantities. 
Firstly, we set the lattice spacing to unity, $a=1$. 
The resulting momenta are therefore given by $(\frac{2\pi n}{N},\frac{2\pi m}{N})$ for $n,m \in [-\frac N2, \frac N2)$.
Furthermore, we express all energy in units of $t$, reducing the number of free parameters in the Hamiltonian therfore by one.
It depends now on only on the parameters $\nicefrac Ut$, $\nicefrac {\mu}t$ and eventually $\nicefrac{t^{\prime}}t$ and $\nicefrac{t^{\prime \prime}}t$.
Temperatures are scaled accordingly with $k_b=1$.
If not otherwise specified, we use only dimensionless units from now on and we will drop $\nicefrac{\cdot7}{t}$ and write just the corresponding symbol, 
assuming it is in units of $t$.


Since the Hubbard model reduces electron-electron interactions in a very crude manner by taking only on-site interactions into account,
its only free parameter $U$ is treated as an adjustable parameter, without an external reference value from other calculations or measurements.	
For this reason the Hubbard model cannot be seen as a first-principle method \cite{J.Phys.Cond.Matter.Vol21.34},
but rather an effective model, that extends a first principle method – the band structure – by introducing one adjustable parameter.
The value of $U$ in materials is not directly observable, but relates to physical quantities as for example the gap in the electrical excitation spectrum.
There are other methods using a Hubbard-like interaction, such as LDA+SO+U calculations.
The $U$ we found can then be compared to the one found through this methods. 
Furthermore it allows us to compare the strength of interaction effects in similar materials in relation to each other.

Numerical calculations were performed in two dimensions due to the layered structure of Sr$_2$IrO$_4$, 
using a grid of size $256\times256$.
As a direct result from the discrete Fourier transformation, we get discrete momenta in the Brillouin zone.
The sum over momenta in Greens functions is therefore restricted to $256\times256$ values.
The magnetic susceptibility $\chi(\vec q,\omega)$ was calculated for different values of $\vec q$.
% path
We choose a path along the symmetry lines of the Brillouin zone, that covers the most interesting features of the dispersion.
Due to symmetry in the band structure energies $\varepsilon_{\vec k}$ and therefore in the energies $E^{\pm}_{\vec k}$ as well,
it is sufficient to restrict oneself to just a quarter of the Brillouin zone.
The most basic dependence on $\vec q$ is through a cosine, $\cos (q_i)$ or $\cos(2q_i)$, that is independent of the sign of the components $q_i$.
%
The path chosen for analysis of $\chi$ is a standard way of presenting calculations and measurements and makes it easy to compare our results to related work.
It covers four directions, which can be parametrized by $t$ ranging from 0 to 1.
First the border of the magnetic Brillouin zone perpendicular to the diagonal, $(\frac14(2-t),\frac14t$, 
then the $k_y$-direction along the border of the Brillouin zone, $(0,\frac12t)$,
the diagonal $(\frac12t,\frac12t)$
and finally 
the $k_x$-direction $(\frac12t,0)$ along the $k_y$-direction.
The path is shown in figure \ref{path}, together with the boundary of the reduced Brillouin zone.
\begin{figure}
 \begin{center}
  \includegraphics[width=.9\textwidth]{../path}
  \caption{path through the Brillouin zone (blue), connecting S-X-M-S-$\Gamma$-X. The grey dotted line is the boundary of the magnetic Brillouin zone.}
 \label{path}
 \end{center}
\end{figure}
It connects the points $\Gamma=(0,0)$, M=($\frac12,\frac12$), S$=(\frac14,\frac14)$ and X=($\frac12,0$) in the order S-X-M-S-$\Gamma$-X.

The temperature was set to $T=0.0033$, corresponding to 10K at t=0.258eV.
This is a temperature  typically encountered in low-temperature experiments and was used in the measurements that provide our reference data. 
The Néel temperature for Sr$_2$IrO$_4$ is at about 250K$^{\text{\cite{PhysRevB.49.9198}}}$.
Being far below the Néel temperature is necessary for our assumption of an anti-ferromagnetic ground state.
A low temperature is also needed in order to be able to describe the material with an one-band Hubbard model. 
Comparison of calculations over a wide range of temperatures up to 200K showed no significant dependence on the temperature except for the broadness of peaks.


In the calculation of the magnetic susceptibility $\chi$ we need to shift the frequencies infinitesimally above the real axis, 
in order to get a retarded response function. 
The small imaginary part of the frequencies are responsible for the imaginary part of $\chi$, which we then relate to the differential cross section. 
In the calculations we used complex frequencies $\omega^+ = \omega + \im \eta$ with a non-zero but small imaginary part $\eta$. 
Choosing a non-infinitesimal value however smears out the delta function, that arises in the limit $\eta \rightarrow 0$ in the imaginary part.
This can be used to compensate for discretisation effects due to the finite sized grid, because we get a contribution even if the delta function 
doesn't match one of the discretised momentum.
% don't repeat yourself here, see derivation of $\chi$ in chapter before.
%
The value has no physical meaning. It changes the broadness and height of the poles. 
It was chosen as small as possible, but big enough to ensure a smooth result for the response function with pronounced features.
Choosing this value to big would wash out the characteristic of the response function, making it hard to analyse.
The frequency dependency of propagators and Greens functions is mostly given by $(\omega^+-E^{\pm}_{\vec k})^{-1}$ and combinations thereof.
As a good starting point we can therefore take the maximum difference in the energies $E^{\pm}$ between to nearby points in the finite sized grid of momenta.
For our simplest dispersion, this occurs for example  between the momenta $\vec k_1 = (\pm \frac14,\pm\frac14)$ and $\vec k_2 = \vec k_1 + (\frac1{\sqrt{N}},0)$.
As an estimation we can take $U=4$ and $m=0.3$ at half filling as a set of typical values, see the discussion below.
We get $|E^{\pm}_{\vec k_1} - E^{\pm}_{\vec k_2}| = 0.001$ for momenta on a $256\times256$ grid.
Finally, we found $\eta=0.005$ to be an optimal value.
This value is of order $10^{-3}$ compared to the maximum $\omega(\vec q)$ encountered for typical values of $U$ used in this thesis.

In order to extract the spin wave dispersion, we calculated the imaginary part and plotted it as 
a colour map over $\vec q$ and $\omega$. 
The colour scale is dominated by the highest peaks of $\chi$.
In order to visualize the whole dispersion we had to choose a cut off for the highest values. 
The value was set individually for each calculation such, that it reveals the whole spectrum of possible excitations,
but such, that the position of the peaks are well defined. 
Areas that were set to the maximal value appear therefore slightly broader in the plot. 
The broadness of the peaks in general is much smaller than in the measurements we compared it to. 

\section{Staggered Magnetization At Half-Filling}
We need to fix the parameters $n_{\sigma}$ and $m_{s,\sigma}$ of the propagators.
They can be expressed through a sum of propagators itself, as shown in equations \ref{n_DEF} and \ref{m_DEF}.
This results in a system of four non-linear coupled equations.
At half filling with $\uparrow$-$\downarrow$-symmetry, they can be decoupled and therby be solved in a easier way. 

The number density is controlled by the chemical potential, which is another free parameter of the system.
The effective spin-$\frac12$ states in Sr$_2$IrO$_4$ are half-filled, that is $n=1$. 
Because of the symmetry between up and down spins we have furthermore $n_{\uparrow}=n_{\downarrow} = \frac12$
This symmetry is not broken but rather assured by the anti-ferromagnetic ground state.

Due to particle-hole symmetry, the Hamiltonian should be invariant under the replacement 
$c_{\vec k,\sigma} \leftrightarrow c_{\vec k,\sigma}^{\dagger}$
% and $\varepsilon_{\vec k} \rightarrow -\varepsilon_{\vec k}$.
In doing this replacement and using the anti-commutation relations for creation and annihilation operators, the Hamilton operator in momentum space reads
\begin{IEEEeqnarray}{rCl}
 \hat H &=& \sum_{\vec k,\sigma} (-\varepsilon_{\vec k} + \mu) c_{\vec k ,\sigma}^{\dagger} c_{\vec k,\sigma} 
 + \frac UN\sum_{\vec k,\vec k^{\prime}, \vec q} c_{\vec k,\uparrow}^{\dagger}c_{\vec k - \vec q,\uparrow}c_{\vec k^{\prime},\downarrow} c_{\vec k^{\prime} + \vec q,\downarrow} 
 \nonumber \\ &&
 -U\sum_{\vec k} \left( c_{\vec k,\uparrow}^{\dagger}c_{\vec k, \uparrow} +c_{\vec k,\downarrow}^{\dagger}c_{\vec k, \downarrow} \right)
  + \sum_{\vec k} (2\varepsilon_{\vec k} + 2\mu + U) 
\end{IEEEeqnarray}
The energy dispersion for holes is given by $\varepsilon_{\vec k}^{\mathrm{holes}} = -\varepsilon_{\vec k}$ 
and by this replacement we get back the original Hamiltonian for particles plus a constant, that does not change our system.
We can collect terms proportional to the total number of holes or particles, 
$nN= \sum_{\vec k,\sigma} c_{\vec k,\sigma}^{\dagger} c_{\vec k,\sigma}$. 
The pre-factor of this terms corresponds to the chemical potential for holes. 
At half-filling it should be the same as for particles, 
we find therefore
\begin{equation}
 \mu = U-\mu \quad \Rightarrow \quad \mu = \frac{U}2.
\end{equation}
This allows us to insert $\mu = \frac U2$ and $n_{\sigma} = \frac12$ immediately in the half filled case.




We furthermore assume that $m_{s,\uparrow}=-m_{s,\downarrow}$. 
The staggered magnetization may thus be expressed by $m_s=2\sigma \cdot m_{s,\sigma}$.
This relation holds in a perfect anti-ferromagnetic state with at most one particle per site. 
Double occupied states on the other hand yield  contribution to $m_{s,\sigma}$, that is symmetric under for $\uparrow \leftrightarrow \downarrow$ in contrast to the 
otherwise asymmetric behaviour.
The expression for $m_{s,\sigma}$ adds $(-1)^i$  whenever a particle with spin $\sigma$ is present at site $i$. 
This is not affected by the simultaneous presence of a particle with spin $-\sigma$, 
which will only be seen by $m_{s,-\sigma}$ and will be weighted with the same factor of $(-1)^i$.
However, double occupancies can be expected to be distributed evenly throughout the system
and we might find as many on the sub-lattice with negative pre-factor as on the positive on.
For large systems we can therefore expect this contributions to vanish in the average over all sites.
This restores the antisymmetric relation between $m_{s,\uparrow}$ and $m_{s,\downarrow}$. 

As a direct result, $F_{\vec k,\sigma}$ depends only by an overall sign on $\sigma$, while the value of $G_{\vec k,\sigma}$ becomes spin independent,
\begin{equation}
 G_{\vec k,\uparrow} = G_{\vec k,\downarrow} \qquad F_{\vec k,\uparrow} =-F_{\vec k,\downarrow}.
\end{equation}




In order to calculate $m_ {s,\sigma}$ we solve equation \ref{m_DEF}, where we can replace $m_{s,\uparrow}$ by $-m_{s,\downarrow}$, as stated above.
We furthermore drop the spin labels on $E_{\vec p,\sigma}^{\pm}$, 
since they depend only on the absolute value of $m_{\sigma}$ and furthermore is $n_{\uparrow} = n_{\downarrow}$ as explained above.
This decouples the two equations with respect to the spin label.
By using the definition of the off-diagonal propagator $F_{\sigma,\vec{p}}$ we end up with
\begin{equation}
 m_{s,\sigma} = \frac1N \sum_{\vec{p}} \sum_{n\in \mathbb{Z}} 
							      \frac { Um_{s,\sigma} }
								    { (\im \omega_n - E_{\vec{p}}^+) (\im \omega_n - E_{\vec{p}})}
\end{equation}
The $m_{s,\sigma}$ on the LHS cancels the one in the RHS, but $E^{\pm}$ is still dependent on $|m_{s,\sigma}|$.
To deal with the summation over the Matsubara frequencies $\im \omega_n$, we use Cauchy's integral theorem as described in \ref{MFS}.
The resulting equation
\begin{equation}
 1= \frac{U}{N}\sum_{\vec{p}} \frac{1}{E_{\vec p}^+ - E_{\vec{p}}^-} \left( \frac{1}{1+\euler^{\beta E_{\vec p}^+}} - \frac{1}{1+\euler^{\beta E_{\vec p}^-}} \right)
\end{equation}
is solved using the Newton-Raphson method.
Figure \ref{ms_nn} shows an example of the behaviour of $m_s$ for different values of $U$. The dispersion in this calculation is based 
on nearest neighbour hopping only, that is $t^{\prime} = t^{\prime \prime} = 0$ on a two-dimensional square lattice.
Introducing non-zero $t^{\prime}$ and $t^{\prime \prime}$ showed the same dependency. 
%
%
\begin{figure}
 \includegraphics[width=.9\textwidth]{../stagmag_T10.png}
 \caption{staggered magnetization as function of $\frac Ut$}
 \label{ms_nn} 
\end{figure}
%
In the limit of infinite $U$ we get $m_s=1$, corresponding to a fully anti-ferromagnetic ground state. 
The strong repulsive interaction prevents double occupancies, because each of them adds $U$ to the energy of the ground state.
We will find therefore at max one particle at each site. Half filling requieres then that there is actually a particle at each site. 
At the same time there are virtual hopping processes to the states of higher energy, which create  a possibility to further lower the ground state energy.
Virtual hopping between neighbouring sites is only possible if they are occupied by particles with opposite spin.
Otherwise it would be forbidden by the Pauli principle, which holds for virtual processes as well. 
Therefore, an anti-ferromagnetic ground state is energetically preferable to a ferromagnetic one.

Lowering $U$ yields a lower staggered magnetization, since the energy cost for double occupancy gets lower and might be out-weighted by
the reduce in kinetic energy due to hopping.
%
At some point we will encounter a transition to an unordered metallic ground state with $m_s=0$. 
The closing of the Hubbard gap leads to the transition from the insulating state to a conducting one.
The critical value for this transition is sensitive to the parameters for second- and third-neighbour hopping, 
even though the dependence of $m_s$ on $U$ above the critical value of $U$ is identical for non-zero and vanishing $t^{\prime}$ and $t^{\prime \prime}$.
The critical value was reported by \citet{PhysRevB.88.035111} to be in the range $U_c \approx 0.7 - 2.2$, depending of the precise set of parameters.
For lower values of $U$, our assumption of a Hubbard gap and an anti-ferromagnetic ground state is not valid any more
and our description is not capable of describing the physical situation any longer. 
The iterative procedure for $m_s$ is not converging in this situation any longer due to the high sensitivity on the band gap through the factor $(E^p_{\vec k} - E^-_{\vec k})^{-1}$.
This term is proportional to $\frac{1}{Um_s}$ for momenta with $\varepsilon_{\vec k} - \varepsilon_{\vec k+\vec Q} =0$.
This is exactly the condition for nesting with nesting vector $\vec Q$, which is present at least at some points.
Calculations down to $U=0.5$ show that the staggered magnetization falls off very fast for low $U$. 
The term $(E^+_{\vec k} -E^-_{\vec k})^{-1}$ diverges in this case and the iteration procedure becomes numerically unstable. 
The limitation for this calculation appears therefore for numerical reasons, that become dominant at roughly the value of the transition to the unordered state.

After calculating $m_{s}$, the expression for $n_{\sigma}$ in terms of the propagator $G_{\vec k,\sigma}$ has also been evaluated 
and found to be consistent with the above condition for the chemical potential at half-filling.



\section{Large U Limit}

A large $U$ drives the system in an anti-ferromagnet ground state as mentioned above. 
Hopping at half-filling leads to double occupancies, which are suppressed due to the high energetic cost of $U$. 
Therefore, the system can be described as an Heisenberg anti-ferromagnet in this limit.
The Heisenberg Hamiltonian with only nearest neighbour coupling is
\begin{equation}
 \hat H = J \sum_{\langle i,j\rangle} \vec{S}_i \cdot \vec{S}_j,
\end{equation}
which will be an anti-ferromagnetic coupling for positive J. 
\todo{derivation of Heisenberg from Hubbard or reference}
Hopping will be treated as a distortion to the system and yields the interaction of neighbouring spins.

%
% http://user.it.uu.se/~ngssc/ngssc_home/S2M2S2/pavarini1.pdf and notes from Morten
% follow explanation, decide in what to include
% include calculation of limit, independent of above considerations
% how does it extend to higher order couplings, that is J^{\prime} and so on as well as t^{\prime}
%

We will expand the expression for $\chi$ in terms of $\frac{t^2}{U}$ 
and compare the resulting dispersion with the linear spin-wave dispersion gained from a $\frac1S$ expansion of the Heisenberg model.
Since we are only interested in the position of the poles in $\chi$, it is sufficient to expand the denominator of equation \ref{ladder_sum} and to solve for its roots,
\begin{equation}
 \left(1+\lambda(x+y)\right)\left(1+\lambda(\bar x + y)\right) - \lambda^2(z_1 +z_2)^2 = 0
\end{equation}
We note first that the term $\lambda y$ is dominant in this limit, since $\lambda y \sim 1$, while $\lambda x \sim \frac1{U^2}$ and $\lambda z_i \sim \frac1U$
for infinite $U$.
Expanding $\lambda y$
up to first order in  $\frac{t^2}U$ results in the dispersion
\begin{equation}\label{DispULimit}
 \omega(\vec q) = \frac{4t^2}{Um_s}\sqrt{4-(\cos q_x+\cos q_y)^2}. 
\end{equation}
This is exactly the spin wave dispersion one gets for linear spin waves in a Heisenberg antiferromagnet, 
with the coupling $J=\frac{4t^2}{Um_s}$.
Deriving the Heisenberg model from the Hubbard model yields $J=\frac{4t^2}{U}$, which differs by a factor of $\frac1{m_s}$ from the coupling we get in this expansion.
The dependence of $m_s$ on $U$ shows, that we get a perfect antiferromagnet with $m_s=1$ for large $U$, 
such that the two couplings will be identical in the case, where this approximations are valid. 
%

Linear spin waves are the first term in an $\frac1S$ expansion of the dispersion in a Heisenberg antiferromagnet.
Higher orders renormalize the spin wave dispersion by allowing for quantum fluctuations.
It was claimed by Peres et al. that the staggered magnetization in the expression for $\omega$ in equation \ref{DispULimit} 
plays the role of this renormalization factor \cite{PhysRevB.65.132404}.
Quantum fluctuations do lower the value of $m_s$, but in the mean field scheme it tends to 1 for large $\frac Ut$ 
and is unable to provide such corrections in the case of large $U$.

The expression for $\omega(\vec q)$ tells us imediatly, that we get a constant dispersion along the 
boundary of the magnetic Brillouin zone, $\omega(\frac12(1-t), \frac12t) =\mathrm{const.}$ for $t \in [0,1]$.
% say sth. why \omega(\vec q = 0) = \omega(\vec q = \vec Q) = 0 is expected. 
% \omega = 0, \vec q = 0: G^2-F^2 = (G-F)^2 - 2GF; (G-F)^2 = 1, GF = z1 = -z2 or similar
% actually is z_1 = z_2 = 0, 1+\lambda(\bar x +y ) = 0 | \vec q =0


By using a large value for $U$ in our calculations, we get results that agree well with the unrenormalized case.
An example for $U=40$ can be seen in figure \ref{largeU}, together with dispersion for linear spin waves according to equation (\ref{DispULimit})ö.
%
\begin{figure}
 \begin{center}
  \includegraphics[width=.8\textwidth]{../U40}
  \caption{spin wave dispersion for $U=40$ with linear spin-wave theory as comparison}
 \label{largeU}
  \end{center}
\end{figure}
%
% higher order terms: t*(t/U)^n, n>1.
% Heisenberg with J-J'-J''
  
\newpage

\section{$t$-$U$-Model}

The $t$-$U$-Model is the simplest version of the Hubbard model.
In dimensionless units it contains $\frac{U}{t}$ as the only free model parameter.
The band structure is then only specified by the geometry of the lattice.
In the case of a square lattice, the band structure fulfils $\varepsilon_{\vec k} = -\varepsilon_{\vec k+\vec Q}$.
By this relation, the energies $E^{\pm}$ of the new bands reduce at half filling to a much simpler form,
\begin{equation}
 E^{\pm}_{\vec k,\sigma} = \pm \sqrt{\varepsilon_{\vec k}^2 -U^2m_{s,\sigma}^2}.
\end{equation}
The energies along the path through the previously defined Brillouin zone are shown in figure \ref{Epmt47}.
The two bands are split by a gap of width $Um_s$. 
They are completely symmetric with respect to the Fermi surface and the energy along the boundary of the reduced Brillouin zone S-X is constant.
%
\begin{figure}
 \centering
 \includegraphics[width=.7\textwidth]{../Edispt47}
 \caption{$E^{\pm}_{\vec q}$ for the $t$-$U$-model with $U=4.7t$. $\vec q$ follows the path in figure \ref{path}.}
 \label{Epmt47}
\end{figure}


Such a model was used by \citet{PhysRevB.65.132404} to calculate the spin wave dispersion of La$_2$CuO$_4$, the cuprate analogue to Sr$_2$IrO$_4$.
They could successfully reproduce the spin wave dispersion measured by inelastic neutron scattering. 
The parameters found were $t_{\mathrm{cup}}=0.295$eV and $U_{\mathrm{cup}}=6.1\cdot t_{\mathrm{cup}}=1.8$eV. 
We use their calculation scheme and adjusted it to the specifications found in Sr$_2$IrO$_4$.
In the first approach we  use the same simple one parameter band structure, based on nearest neighbour hopping only.

The spin wave dispersions of both materials show a very similar qualitative behaviour.
The most characteristic difference is the dispersion along the boundary of the reduced Brillouin zone, e.g. the line S-X in figure \ref{path}.
The ratio of frequencies at S and X,
\begin{equation} 
r = \frac{ \omega(\frac14,\frac14) }{ \omega(\frac12,0) }. 
\end{equation}
provides a quantitative measure of this effect. 
In  La$_2$CuO$_4$ the experimentally measured ratio is relatively large with an value of $r = 0.85$, compared to $r=0.54$ in
Sr$_2$IrO$_4$.

The ratio $r$ increases with $U$ and reaches eventually $r=1$ 
in the Heisenberg limit, where 
 the dispersion is constant along the line S-X.
 
The Heisenberg model with nearest neighbour couplings corresponds to an expansion up to first order in $\frac tU$ only, 
i.e. only terms proportional to $\frac {t^2}{U}$ are taken into account.
As we lower the interaction strength, higher order terms become more important.
The Heisenberg model can give a lower value for $r$ only if interactions between further neighbours are introduced.
They arise from  hopping processes over multiple sites, which 
gives  a prefactor proportional to $t\cdot(\frac tU)^n$  for $n+1$ hopping processes.
Expanding to third order yields the Heisenberg Hamiltonian with further neighbour coupling and ring exchange, namely
\begin{IEEEeqnarray}{rCl}
 \hat H &=& J\sum_{\langle i,j \rangle }  \vec S_i \cdot \vec S_j 
    + J^{\prime} \sum_{\langle \langle i,j\rangle \rangle} \vec S_i \cdot \vec S_j  \nonumber \\ &&
    + J^{\prime} \sum_{\langle\langle \langle i,j\rangle \rangle \rangle} \vec S_i \cdot \vec S_j \nonumber \\ &&
    + J_c \sum_{\langle i,j,k,l \rangle} \left( (\vec S_i \cdot \vec S_j)(\vec S_k \cdot \vec S_k)  + (\vec S_i \cdot \vec S_l)(\vec S_k\cdot \vec S_j)
    -(\vec S_i \cdot \vec S_k) (\vec S_j \cdot \vec S_l)\right).
\end{IEEEeqnarray}
The couplings are given by \cite{0022-3719-10-8-031} 
\begin{equation}
 J = 4\frac{t^2}{U} -24\frac{t^4}{U^3}; \qquad J^{\prime} = J^{\prime \prime} = 4\frac{t^4}{U^3}; \qquad J_c = 80 \frac{t^4}{U^3}.
\end{equation}
Primes denote second and third neighbours respectively.
The index $c$ stands for cyclic exchange, 
i.e. hopping of a particle over four sites such that it ends up at the site where it started. 
where the labels $\langle i,j,k,l\rangle$ in the last sum denotes a ring of four sites, that are labelled anticlockwise. 
Especially the ring exchange seems to contribute to the dispersion along the zone boundary \cite{PhysRevLett.86.5377}.
It can be absorbed in the neighbour exchange couplings. In the spin-$\frac12$ case the relations are
\begin{equation}
 J_{\mathrm{eff}} = J -\frac12J_c, \qquad  J^{\prime}_{\mathrm{eff}} = J^{\prime} - \frac14 J_c,
\end{equation}
while the third neighbour exchange $J^{\prime \prime}$ remains unchanged.
%
% absorb J_c into J-J-J
%


We found that the Hubbard model based on nearest neighbour hopping only is not capable of describing the dispersion observed in experiments.
Choosing $U=4.7 = 1.2$eV provides the best description of the spin wave dispersion along most parts of the path.
A comparison with measurements can be seen in figure \ref{tU47}. 
%
\begin{figure}
 \begin{center}
  \includegraphics[width=.8\textwidth]{../compare_t47}
  \caption{Renormalized spin wave dispersion for the $t$-$U$-model, $U=4.7$ (blue). Graphs for RIXS measurements (red) and $J$-$J^{\prime}$-$J^{\prime \prime}$-fit (green) are
	    token from \cite{PhysRevLett.106.136402} and were recoloured. The exchange couplings $J$, $J^{\prime}$ and $J^7{\prime \prime}$ are treated there as independent 
	    parameter and do not follow the relations to $U$ and $t$.}
 \label{tU47}
 \end{center}
\end{figure}
%
The spin wave velocity at $c=\left. \frac{\dint \omega}{\dint \vec k} \right|_{\vec k = 0}$ is well described after renormalization as described in chapter \ref{corrections}.
With the same renormalization factor the dispersion along
the zone boundary of the whole Brillouin zone, namely from X to M, is represented at a high level of agreement.
The model fails however to catch the lowered dispersion around $\vec q = (\frac 14, \frac14)$.
The calculated dispersion is way to high, giving us a ratio of $r=0.8$. 
It is possible to get to lower values of $r$ for smaller $U$, but the resulting dispersion fails then to match the spin wave velocity and 
the total energy scale. An value for $r$ as low as the one found in the iridates will not be reached in this model
for any value of $U$ above the critical value of an metal-insulator transition.



In addition to the excitations, where $\omega$ has a clear functional dependence on $\vec q$, 
we find an area of continuous excitations, as can be seen in figure \ref{Xcomponents}.
We show the contributions of the longitudinal ($\chi^{zz}$) and transversal ($\chi^{+-}$) component of the susceptibility separately.
This reveals 
that the spin wave dispersion at lower energies is determined by $\chi^{+-}$ completely.
\begin{figure}
 \centering
 \begin{subfigure}{.49\linewidth}
\includegraphics[width=\textwidth]{../tXpm.png}
  \caption{transversal susceptibility $\chi^{+-}$}
 \end{subfigure}
\begin{subfigure}{.49\linewidth}
 \includegraphics[width=\textwidth]{../tXzz.png}
 \caption{longitudinal susceptibility $\chi^{zz}$}
\end{subfigure}
\caption{Transversal and longitudinal component of $\chi$ in the $t$-$U$-model for $U=4.7t$, including the continuum at higher energies $\omega$. 
	 The intensity was truncated at the same value for both pictures and such, that the lower intensities of the continuum are clearly visible.}
	 \label{Xcomponents}
\end{figure}
%
while contributions from $\chi^{zz}$ to the excitation spectrum lie solely in the continuum, as can be seen from.
Excitations in the longitudinal direction are only possible, when the anti-ferromagnetic ordering of the ground state 
is distorted, which comes with a energetic cost, that increases with $U$.

The continuum begins at energies, comparable to the band gap between $E^+_{\vec k}$ and $E^-_{\vec k}$ and 
stretches mainly over an area of two times the band width of $E^{\pm}_{\vec k}$.
for the $t$-$U$-model the continuum begins at 0.54eV for $U=4.7$.
Around the critical value $U_c$, continuum excitations merge with the spin wave excitations. 
In experiments a large gap is found between the low energy excitations and the continuum, which indicates a $U$ well above the critical value \cite{PhysRevLett.106.136402}. 

This continuum corresponds to the excitation of an electron into the otherwise unoccupied upper energy band $E^+_{\vec k}$.
The emergence of the continuum can be inferred from the  mathematical expression of $\chi$. 
In calculating bubble diagrams, one encounters sums over terms like $(E^{-}_{\vec p} +\omega - E^+_{\vec p + \vec q})^{-1}$.
They are suppressed by a factor of $(1+\euler^{E^-_{\vec k} T})^{-1}$, but
for $\omega$ larger than the band gap, but lower than the maximum difference between the bands, their denominator vanishes for certain $\omega$ 
and they can give a big contribution to the sum over such expressions for different momenta. 
In experiment the measured continuum begins around 0.5eV \cite{PhysRevLett.86.5377}, which is of the same magnitude, but less then the lower bound found in our calculations based
on the $t$-$U$-model.
Furthermore, the lower bound of continuous excitations is not a straight line as it is in our calculations. 
This indicates that the model is to much simplified and missing therefore some essential features.
At these energy scales there are further excitations possible, for example excitations to the $J=\frac32$ states, which are not covered by our calculation scheme.
%The effect of the canted structure can be seen by the small intensities around the point M, that  shifted  from $\Gamma$ with 
%a weight of $\frac{\sin \theta}{\cos \Theta}$.





% explanation for the gap? Need excitations in the upper band cor \chi^zz?
% What happens afterwards?

% compare to measurements: Seen there as well, but  a little bit to high and sharp
% doesnt fit to the band gap, that is overestimated anyway. 






% continuum,
% continuum in the case of very low $U$
 
% effect of t' and t'' clearer around FS. 
% more influnce...

%%%%%%%%%%%%%%%%%%%%%%%%%%%%%%%%%%%%%%%%%%%%%%%%%%%%%%%%%
%
% could elaborate on calculation details of x,y,z_i.
% e.g. Wick rotation, Matsubara frequency summation etc. 
% and how the final terms actually count, in order to justify the arguments made later.
%
%%%%%%%%%%%%%%%%%%%%%%%%%%%%%%%%%%%%%%%%%%%%%%%%%%%%%%%%%



% t= 0.258eV, Temperature
%We used the same $t$ as in the previous situation without second- and third neighbour couplings, i.e. $t^{\prime} = t^{\prime \prime} = 0$, 
%when converting values like the temperature.


\section{$t$-$t^{\prime}$-$t^{\prime \prime}$-$U$-Model}

In the previous chapter we saw that an approach with only nearest neighbour interactions  is not capable of reproducing all the observed features 
in the magnetic excitations of  Sr$_2$IrO$_4$.
In order to improve our model, we extend the band structure beyond nearest neighbour hopping by introducing second and third neighbour hopping terms.
This provides a more realistic description of the system and even though the parameters are small compared to $t$, 
they are needed in order to correctly reproduce the measured spin wave dispersion.


The hopping parameters $t,t^{\prime},t^{\prime \prime}$ were treated as fixed external parameters.
This means they were taken from first principle calculations done by 
\citet{PhysRevLett.106.136402} rather than adjusted to experiments.
We use again $t$ as the basic  energy unit and express other parameters in terms of it.
There are therefore two new parameters entering the calculations, the ratios $\nicefrac{t^{\prime}}{t}$ and $\nicefrac{t^{\prime \prime}}{t}$, 
which will be denoted by $t^{\prime}$ and 
$t^{\prime \prime}$ in the following discussion. 
Normalizing the energy scale to $t$ reduces not only the number of parameters, it is also easier to find the ratio of neighbour interactions from calculations.
In order to get  absolute values, one needs to adjust the energy scale of the resulting dispersion to a known absolute value of the band \cite{PhysRevB.67.064504}.


The values from \cite{PhysRevLett.106.136402} are based on fits to a multi orbital tight binding model that include all the $t_{2g}$ states 
together with the hybridization with the $p$ orbitals from the ligands.
The couplings between these orbitals were then fitted to reproduce the band dispersion that was calculated in the linear density approximation (LDA) including SOC. 
After a projection to the spin-$\frac12$ subspace, we get the effective couplings of between sites in our one orbital Hubbard model.
The values used in this thesis were 
%
\begin{table}
\begin{center}
\begin{tabular}{|c|c|c|}
\hline
$t$ & $t^{\prime}$ & $t^{\prime \prime}$ \\
\hline
0.258eV & 0.06eV & 0.03eV \\
1 & 0.2326 & 0.1163 \\
\hline
\end{tabular}
\caption{Hopping parameters for the band structure in the $t$-$t^{\prime}$-$t^{\prime\prime}$-$U$-model \cite{PhysRevLett.106.136402}.
The second line contains the relative values, normalized to $t$. }
\label{ttt}
\end{center} 
\end{table}
%
With those parameters the band structure of the square lattice is 
\begin{equation}
  \varepsilon_{\vec k } = -2t \left(\cos k_x + \cos k_y \right) -4t^{\prime} \cos k_x \cos k_y  -2t^{\prime \prime} \left( \cos 2k_x + \cos 2k_y \right)
\end{equation}
as explained in chapter \ref{chapter_bandstructure}.



The second and third neighbour hopping parameters are substantially smaller than $t$, but big enough to change the 
results significantly. 
 The most important feature is the removal of the degeneracy in $E^{\pm}_{\vec k}$ along the border of the reduced Brillouin zone.
This was the main reason, why the $t$-$U$-model was not capable of reproducing the spin wave dispersion. 
In figure \ref{tttE42} we plotted again the energies of $E^{\pm}_{\vec q}$ along the same path as before.
\begin{figure}
 \centering
\includegraphics[width=.8\textwidth]{../Edispttt44}
 \caption{$E^{\pm}_{\vec q}$ in the extended model that includes second and third neighbour hopping terms.
 $U$ was chosen to be 4.4$t$,
 $\vec q$ follows the path through the Brillouin zone described in figure \ref{path}. }
 \label{tttE42}
\end{figure}
The gap between the upper and lower band is smaller than in the previous case, where it was given by $Um_s$.
At the same time the bands stretch over a larger energy range.





The $t$-$t^{\prime}$-$t^{\prime \prime}$-$U$-Hubbard model
is capable of describing all features of the measured dispersion to a high level of agreement. 
The optimal value for the Hubbard interaction was found to be $U=4.4=1.1eV$.
The value was optimized in several trials by guidance of the eye, until the experimental values were met.
With the renormalization due to quantum fluctuations a very high level of agreement could be achieved, see figure \ref{tttU42}.
%
\begin{figure}
 \begin{center}
  \includegraphics[width=\textwidth]{../tttU44c.png}
 \caption{ Spin-wave dispersion for $\frac Ut=4.4$ for the $t$-$t^{\prime}$-$t^{\prime}$-$U$-Hubbard model (blue) 
 compared to measurements (red) and a up to third neighbour Heisenberg model (green). 
 Experimental values and Heisenberg fit are taken from \cite{PhysRevLett.108.177003} and recoloured.} 
\label{tttU42}
 \end{center}
\end{figure}
%
Again, the spin wave velocity and the dispersion along X-M and $\Gamma$-X is very well described. 
The ratio $r$ of energies at $\vec q = (\frac14,\frac14)$ and $(\frac12,0)$ is slightly overestimated.
At $\vec q = (\frac14,\frac14)$ and in its direct vicinity the spin-wave energy is a little overestimated.
This results in a larger value of $r$, since $\omega(\frac12,0)$ is matched exactly.
We get $r=$ which overestimates the measured value by 5\%.
The derivation is small compared  to the uncertainties in the measured spin-wave energies. 



The staggered magnetization at $U=4.4$ is $m_s= 0.73$ according to Eq. (\ref{m_DEF}).
As mentioned above, there is a net ferromagnetic moment proportional to the anti-ferromagnetic ordering 
parameter $m_s$, because of the canted orientation of the oxygen octahedra. 
We found for the magnetic ordering parameter $m = m_s \sin \Theta = 0.139$. 
It is related to the total magnetic moment by the magnetic moment of a single site, that is defined as $g\mu_b J$.
The observed ferromagnetic moment is $0.14\mu_b$ \cite{PhysRevB.57.R11039}. 
Our result is therefore in good agreement, given the magnetic moment of $1\mu_b$ at each Ir site, which is the value of the atomic limit.
The canted structure provides therefore a good explanation of the observed weak ferromagnetic moment 
in an anti-ferromagnetic ground state.

Another quantity, that is closely related to the Hubbard-$U$ is the width of the Mott gap.
The Mott gap is the split of the $J=\frac12$-band due to the repulsive Hubbard interaction into the bands denoted by $E^{\pm}_{\vec k}$ respectively. 
With the optimal value of $U$ found in  our calculation and the corresponding staggered magnetization $m_s$, 
the width of the band gap is given by $\Delta_{\mathrm{Mott}}=1.6t=0.47\mathrm{eV}$.
This is close to, but slightly lower than the experimental result of 0.54eV  at $T=10$K,
as found through optical spectroscopic measurements by \citet{PhysRevB.80.195110}.


With the changed dispersion the continuous excitations at higher energies agree with experiments on a qualitative level. 
The calculated dispersion is compared to the measured one in \ref{continuum}.
\begin{figure}
 \centering
 \begin{subfigure}{0.49\linewidth}
  \includegraphics[width=\textwidth]{../tttU44_longw}
  \caption{calculated spin wave dispersion for $U=4.4$.}
  \label{tttU44_longw}
 \end{subfigure}
\begin{subfigure}{.49\linewidth}
 \includegraphics[width=\textwidth]{../measuredexit}
 \caption{measured excitations, taken from \cite{PhysRevLett.108.177003} }
 \label{experimental_longw}
\end{subfigure}
\caption{comparison of the excitation spectrum  form our calculations and RIXS measurements. The energy scale in both graphs is given in eV. 
	  The measured excitations contain also spin orbit excitations between 0.4eV and 0.8eV}
\label{continuum}	  
\end{figure}
Due to the changed energy bands, our result shows no longer a constant boundary at the lower end, but displays $\vec k$-dependence as well.
A quantitative comparison of the excitations is difficult, since the experimental data contains  spin orbit excitations as well. Those lie in the centre 
of the continuous band.
On a qualitative level we note that the bulges at the low energy end of the continuum  appear for the same momenta as in the measurements. 
The lowest excitations are found at $\vec q = (\frac14,\frac14)$. This is the momentum transfer that connects the $\vec k$ for the highest value of $E^-_{\vec k}$ with the 
momentum, where $E^+_{\vec k}$ is minimal, as one would expect for such excitations.




% compare to J-J-J and explain why it is better
% look up J in t-t'-t''-U expansion




\citet{PhysRevLett.108.177003} did not only measure the inelastic scattering cross section, they provided also a model to describe the data.
Their model is the Heisenberg model with up to third neighbour couplings, with the parameters $J$, $J^{\prime}$ and $J^{\prime \prime}$ for
the first, second and third neighbour exchange couplings respectively, without an parameter for cyclic exchange. 
All three parameters were treated as free parameters and fitted to the measured spin wave dispersion. 
The values were found to be $J=60$meV, $J^{\prime}=-20$meV and $J^{\prime\prime}=15$meV. 
The resulting dispersion reproduces the right energies at the points X and S and therefore the right $r$ as well,
while the energies along the path connecting these points are systematically underestimated.



In an effective model the effect of cyclic exchange can be absorbed in the other parameters.
$J$ and $J^{\prime}$ will be lowered by that, while $J^{\prime \prime}$ remains unchanged. 
As an expansion of the Hubbard model, these parameters are related through their dependence on $t,t^{\prime},t^{\prime \prime}$ and $U$.
Up to second order, they are given by
\begin{IEEEeqnarray}{ccc}
 J= 4\frac{t^2}{U} - 64 \frac{t^4}{U^3} = 41\mathrm{meV} &
 ,\qquad J^{\prime} =4\frac{t^{\prime 2}}{U}-16\frac{t^4}{U^3} = -38\mathrm{meV} \nonumber \\
  J^{\prime \prime}= 4\frac{t^{\prime \prime 2}}{U}+  4\frac{t^3}{U^4} = 15\mathrm{meV}.&
\end{IEEEeqnarray}
where we have inserted the values found above for $U$ and $t,t^{\prime},t^{\prime \prime}$ from table \ref{ttt}.
These values do not agree very well with the fit and show that their parameters do not follow the functional dependence on the parameters of the Hubbard model.
Also our Hubbard model fit describes the dispersion with a higher accuracy. 
At the same time it has just one free parameter, while the other parameters can be obtained by first principle calculations.


% interpret value of U, compare to cuprates and literature: lower than usually 


Our optimal value for $U=4.4$ corresponding to 1.1eV is substantially lower than the one found in the similar cuprate compounds.
In the cuprates $U$ is usually set to around 7$t_{\mathrm{Cu}}$ for the corresponding $t_{\mathrm{Cu}}=0.3$eV \cite{PhysRevB.65.132404}.
This is expected, since the $5d$ orbitals are more extended, which reduces the intra orbital interaction.
In the literature the Hubbard interaction is often estimated to be $U=2$eV in iridates \cite{PhysRevB.80.075112}. 
Our calculations shows that the interaction is weaker and $U=1.1$eV yields a more realistic description of the system.
With this value of $U$ and its SOC strength, \Sriro is still in the anti-ferromagnetic phase, but close to the transition point to a paramagentic metal \cite{PhysRevLett.105.216410}. 






\section{Outlook}

We showed using the example of Sr$_2$IrO$_4$ that the one band Hubbard model in the mean field treatment works well as an effective model
for correlated systems with strong spin orbit coupling. 
Using previously calculated band structure parameters we introduced the Hubbard interaction and fixed its strength, the only free parameter in the model, by comparison to experiment. 
We could show how hopping terms beyond the nearest neighbours are necessary to describe the observed dispersion. 
The model reproduced the spin wave dispersion very well after introducing a constant renormalization factor. 
Furthermore, it gave a realistic sketch of the continuum at higher energies.
We found a value of $U=1.1eV$, which is lower than the one usually assumed in the literature.
The value is close to the metal-insulator transition point, but clearly in the insulating phase.
Furthermore the model provides an explanation for the value of the charge gap and yields the correct ferromagnetic moment of this material.


There is room for improvement in the analysis of the susceptibility. 
One could take not only the position of the poles into account, but also quantify their height and broadness, making a more precise comparison to experiment possible. 
We found corrections due to quantum fluctuations necessary to include and that the most important contribution of spin wave velocity renormalization is momentum independent. 
It might however be worth to develop a more elaborate way of calculating corrections in the Hubbard model itself refining thereby the Greens functions itself. 

In further work this model can be extended to other crystals with anti-ferromagnetic ordering
like the other elements of the Ruddelson-Popper series, namely SrIrO$_3$ and the bilayer iridate Sr$_3$Ir$_2$O$_7$.
The different geometric set-up gives a different band structure due to changed geometry and hopping parameters. 
It can be further improved by introducing hopping between layers, which will be more important in multi-layered iridates. 

Another interesting application for this type of calculation is the doped case, e.g. a filling factor that is shifted away from $n=1$. 
It is yet an open question, if iridates can be doped such that they show superconductivity as found in the cuprates.






